\documentclass[12pt]{book}
\setlength{\paperheight}{11in}
\setlength{\paperwidth}{8.5in}
\setlength{\topmargin}{0pt}
\setlength{\voffset}{-28.908pt}
\setlength{\evensidemargin}{36.135000000000005pt}
\setlength{\oddsidemargin}{36.135000000000005pt}
\setlength{\textwidth}{433.62pt}
\setlength{\textheight}{645.0097499999999pt}
\setlength{\headheight}{14.5pt}
\setlength{\headsep}{13.10125pt}
\setlength{\footskip}{.5in}
\DeclareTextSymbol{\textsquarebracketleft}{EU1}{91}
\DeclareTextSymbol{\textsquarebracketright}{EU1}{93}
\usepackage{needspace}
\usepackage{xltxtra}
\usepackage{setspace}
\usepackage[normalem]{ulem}
\usepackage{color}
\usepackage{colortbl}
\usepackage{tabularx}
\usepackage{longtable}
\usepackage{multirow}
\usepackage{booktabs}
\usepackage{calc}
\usepackage{fancyhdr}
\usepackage{fontspec}
\usepackage[pdfauthor={Hugh J. Paterson III}, pdfcreator={XLingPaper version 2.25.0 (www.xlingpaper.org)}, pdftitle={Keyboard Layout Design for Minority Languages - (Socio)linguistic (app/im)plications}]{hyperref}
\hypersetup{colorlinks=true, citecolor=black, filecolor=black, linkcolor=black, urlcolor=blue, bookmarksopen=true}
\fancypagestyle{frontmattertitle}
{\fancyhf{}
\renewcommand{\headrulewidth}{0pt}
\renewcommand{\footrulewidth}{0pt}
}\fancypagestyle{frontmatterfirstpage}
{\fancyhf{}
\fancyfoot[C]{{\XLingPaperTimesZNewZRomanFontFamily{\thepage}}}
\renewcommand{\headrulewidth}{0pt}
\renewcommand{\footrulewidth}{0pt}
}\fancypagestyle{frontmatter}
{\fancyhf{}
\fancyfoot[CE]{{\XLingPaperTimesZNewZRomanFontFamily{\thepage}}}
\fancyfoot[CO]{{\XLingPaperTimesZNewZRomanFontFamily{\thepage}}}
\renewcommand{\headrulewidth}{0pt}
\renewcommand{\footrulewidth}{0pt}
}\fancypagestyle{bodyfirstpage}
{\fancyhf{}
\fancyfoot[C]{{\XLingPaperTimesZNewZRomanFontFamily{\thepage}}}
\renewcommand{\headrulewidth}{0pt}
\renewcommand{\footrulewidth}{0pt}
}\fancypagestyle{body}
{\fancyhf{}
\fancyfoot[CE]{{\XLingPaperTimesZNewZRomanFontFamily{\thepage}}}
\fancyfoot[CO]{{\XLingPaperTimesZNewZRomanFontFamily{\thepage}}}
\renewcommand{\headrulewidth}{0pt}
\renewcommand{\footrulewidth}{0pt}
}\setmainfont{Times New Roman}
\font\MainFont="Times New Roman" at 12pt
\newfontfamily{\XLingPaperCharisZSILZSmallZCapsFontFamily}{Charis SIL Small Caps}
\newfontfamily{\XLingPaperTimesZNewZRomanFontFamily}{Times New Roman}
\setlength{\parindent}{.3in}
\doublespacing
\catcode`^^^^200b=\active
\def^^^^200b{\hskip0pt}
\let\origdoublepage\cleardoublepage
\newcommand{\clearemptydoublepage}{\clearpage{\pagestyle{empty}\origdoublepage}}\renewenvironment{quotation}{\list{}{\leftmargin=.25in\rightmargin=.25in}\item[]{}}{\endlist}
\clubpenalty=10000
\widowpenalty=10000
\begin{document}
\baselineskip=\glueexpr\baselineskip + 0pt plus 2pt minus 1pt\relax
\renewcommand{\footnotesize}{\fontsize{10}{12}\selectfont }
\newlength{\leveloneindent}
\newlength{\levelonewidth}
\newlength{\leveltwoindent}
\newlength{\leveltwowidth}
\newlength{\levelthreeindent}
\newlength{\levelthreewidth}
\newlength{\levelfourindent}
\newlength{\levelfourwidth}
\newlength{\levelfiveindent}
\newlength{\levelfivewidth}
\newlength{\levelsixindent}
\newlength{\levelsixwidth}
\newdimen\XLingPapertempdim
                \newdimen\XLingPapertempdimletter
                \newcommand{\XLingPapertableofcontents}{\immediate\openout8 = \jobname.toc\relax
\write8{<toc>}}
\newcommand{\XLingPaperaddtocontents}[1]{\write8{<tocline ref="#1" page="\thepage"/>}}
\newcommand{\XLingPaperendtableofcontents}{\write8{</toc>}\closeout8\relax
}
\newcommand{\XLingPaperdotfill}{\leaders\hbox{$\mathsurround 0pt\mkern 4.5 mu\hbox{.}\mkern 4.5 mu$}\hfill}
\newcommand{\XLingPaperdottedtocline}[4]{
\newdimen\XLingPapertempdim
\vskip0pt plus .2pt{
\leftskip#1\relax% left glue for indent
\rightskip\XLingPapertocrmarg% right glue for for right margin
\parfillskip-\rightskip% so can go into margin if need be???
\parindent#1\relax
\interlinepenalty10000
\leavevmode
\XLingPapertempdim#2\relax% numwidth
\advance\leftskip\XLingPapertempdim\null\nobreak\hskip-\leftskip{#3}\nobreak
\XLingPaperdotfill\nobreak
\hbox to\XLingPaperpnumwidth{\hfil\normalfont\normalcolor#4}
\par}}
\newlength{\XLingPaperpnumwidth}
\newlength{\XLingPapertocrmarg}
\setlength{\XLingPaperpnumwidth}{1.55em}\setlength{\XLingPapertocrmarg}{\XLingPaperpnumwidth+1em}
\newlength{\XLingPaperlistinexampleindent}
\newlength{\XLingPaperisocodewidth}\setlength{\XLingPaperlistinexampleindent}{0in+ 2.75em}
\newlength{\XLingPaperlistitemindent}
\newlength{\XLingPaperbulletlistitemwidth}\settowidth{\XLingPaperbulletlistitemwidth}{•\ }\newlength{\XLingPapersingledigitlistitemwidth}
\settowidth{\XLingPapersingledigitlistitemwidth}{8.\ }\newlength{\XLingPaperdoubledigitlistitemwidth}
\settowidth{\XLingPaperdoubledigitlistitemwidth}{88.\ }\newlength{\XLingPapertripledigitlistitemwidth}
\settowidth{\XLingPapertripledigitlistitemwidth}{888.\ }\newlength{\XLingPapersingleletterlistitemwidth}
\settowidth{\XLingPapersingleletterlistitemwidth}{m.\ }\newlength{\XLingPaperdoubleletterlistitemwidth}
\settowidth{\XLingPaperdoubleletterlistitemwidth}{mm.\ }\newlength{\XLingPapertripleletterlistitemwidth}
\settowidth{\XLingPapertripleletterlistitemwidth}{mmm.\ }\newlength{\XLingPaperromanviilistitemwidth}
\settowidth{\XLingPaperromanviilistitemwidth}{vii.\ }\newlength{\XLingPaperromanviiilistitemwidth}
\settowidth{\XLingPaperromanviiilistitemwidth}{viii.\ }\newlength{\XLingPaperromanxviiilistitemwidth}
\settowidth{\XLingPaperromanxviiilistitemwidth}{xviii.\ }\newlength{\XLingPaperspacewidth}
\settowidth{\XLingPaperspacewidth}{\ }\newlength{\XLingPapersingledigitlistofwidth}
\settowidth{\XLingPapersingledigitlistofwidth}{8.  }\newlength{\XLingPaperdoubledigitlistofwidth}
\settowidth{\XLingPaperdoubledigitlistofwidth}{88.  }\newlength{\XLingPapertripledigitlistofwidth}
\settowidth{\XLingPapertripledigitlistofwidth}{888.  }\newcommand{\XLingPaperlistitem}[4]{
\newdimen\XLingPapertempdim
\vskip0pt plus .2pt{
\leftskip#1\relax% left glue for indent
\parindent#1\relax
\interlinepenalty10000
\leavevmode
\XLingPapertempdim#2\relax% label width
\advance\leftskip\XLingPapertempdim\null\nobreak\hskip-\leftskip\hbox to\XLingPapertempdim{\hfil\normalfont\normalcolor#3\ }{#4}\nobreak
\par}}
\newcommand{\XLingPaperblockquote}[4]{\vskip#3{
\leftskip#1\relax% left glue for indent\relax
\rightskip#1\relax% right glue for indent
\interlinepenalty10000
\leavevmode\hskip-\parindent{#2}\nobreak
}\vskip#4}
\newcommand{\XLingPaperexample}[5]{
\newdimen\XLingPapertempdim
\vskip0pt plus .2pt{
\leftskip#1\relax% left glue for indent
\hspace*{#1}\relax
\rightskip#2\relax% right glue for indent
\interlinepenalty10000
\leavevmode
\XLingPapertempdim#3\relax% example number width
\advance\leftskip\XLingPapertempdim\null\nobreak\hskip-\leftskip\hbox to\XLingPapertempdim{\normalfont\normalcolor#4\hfil}{#5}\nobreak
\par}}
\newcommand{\XLingPaperexampleintable}[5]{
\newdimen\XLingPapertempdim
\leftskip#1\relax% left glue for indent
\hspace*{#1}\relax
\rightskip#2\relax% right glue for indent
\interlinepenalty10000
\leavevmode
\XLingPapertempdim#3\relax% example number width
\hbox to\XLingPapertempdim{\normalfont\normalcolor#4\hfil}{
\begin{tabular}
[t]{@{}l@{}}#5\end{tabular}
}\nobreak
}
\newcommand{\XLingPaperfree}[2]{\vskip0pt plus .2pt{
\leftskip#1\relax% left glue for indent
\parindent#1\relax
\interlinepenalty10000
\leavevmode{#2}\nobreak
\par}}
\newcommand{\XLingPaperlistinterlinear}[5]{\vskip0pt plus .2pt{\hspace*{#1}\hspace*{#2}
\XLingPapertempdimletter#3\relax% letter width
\advance\leftskip\XLingPapertempdimletter\null\nobreak\hskip-\leftskip\hspace*{-.3em}\hbox to\XLingPapertempdimletter{\normalfont\normalcolor#4\ \hfil}{#5}\nobreak
\par}}
\newcommand{\XLingPaperlistinterlinearintable}[5]{
\XLingPapertempdimletter#3\relax% letter width
\hspace*{-.3em}\hbox to\XLingPapertempdimletter{\normalfont\normalcolor#4\ \hfil}{
\begin{tabular}
[t]{@{}l@{}}#5\end{tabular}
}\nobreak
}

\newlength{\XLingPaperexamplefreeindent}\setlength{\XLingPaperexamplefreeindent}{-.3 em}\newskip\XLingPaperinterwordskip
\XLingPaperinterwordskip=6.66666pt plus 3.33333pt minus 2.22222pt
\def\XLingPaperintspace{\hskip\XLingPaperinterwordskip}
\def\XLingPaperraggedright{\rightskip=0pt plus1fil\pretolerance=10000}\raggedbottom
\pagestyle{fancy}
\begin{MainFont}
\XLingPapertableofcontents\pagenumbering{roman}
\pagestyle{frontmattertitle}\pagestyle{frontmattertitle}{\clearpage
\vspace*{1.25in}\noindent
\fontsize{14}{16.8}\selectfont \textbf{{\centering
Keyboard Layout Design for Minority Languages - (Socio)linguistic (app/im)plications\\}}}\par{}
{\clearpage
\vspace*{1.25in}\noindent
\fontsize{14}{16.8}\selectfont \textbf{{\centering
Keyboard Layout Design for Minority Languages - (Socio)linguistic (app/im)plications\\}}}\par{}
{\noindent
\textit{{\centering
Hugh J. Paterson III\\}}}\par{}
{\noindent
\textit{{\centering
SIL International\\}}}\par{}
{hugh@thejourneyler.org}\par{}
{\noindent
\fontsize{10}{12}\selectfont {\centering
04. April 2015\\}}\par{}
{\noindent
\fontsize{10}{12}\selectfont {\centering
Version: Version: 0.7\\}}\par{}
\clearpage
\pagestyle{frontmatter}\thispagestyle{frontmatterfirstpage}\clearpage
\thispagestyle{frontmatterfirstpage}{\needspace{3\baselineskip}
\vspace*{.65in}\noindent
\raisebox{\baselineskip}[0pt]{\pdfbookmark[1]{Contents}{rXLingPapContents}}\raisebox{\baselineskip}[0pt]{\protect\hypertarget{rXLingPapContents}{}}\noindent
{\MakeUppercase{{\protect\centering
Contents\\}}}\markboth{Contents}{Contents}
\XLingPaperaddtocontents{rXLingPapContents}}\penalty10000\par{}
\vspace{10.8pt}{\singlespacing
\hyperlink{rXLingPapPreface1}{\XLingPaperdottedtocline{0pt}{0pt}{List of tables}{vi}
}}{\singlespacing
\hyperlink{rXLingPapPreface2}{\XLingPaperdottedtocline{0pt}{0pt}{List of figures}{vii}
}}{\singlespacing
\hyperlink{rXLingPapPreface3}{\XLingPaperdottedtocline{0pt}{0pt}{List of abbreviations}{viii}
}}{\singlespacing
\hyperlink{rXLingPapAcknowledgements}{\XLingPaperdottedtocline{0pt}{0pt}{Acknowledgements}{ix}
}}{\singlespacing
\hyperlink{rXLingPapAbstract}{\XLingPaperdottedtocline{0pt}{0pt}{Abstract}{x}
}}{\singlespacing{}\noindent{}CHAPTER\par{}}\settowidth{\levelonewidth}{1. \ }{\singlespacing
\hyperlink{c}{\XLingPaperdottedtocline{.3in}{\levelonewidth}{1. \ Introduction}{1}
}}\settowidth{\leveltwoindent}{{1. }\ }\settowidth{\leveltwowidth}{{1.1. }\thinspace\thinspace}\addtolength{\leveltwoindent}{.3in}{\singlespacing
\hyperlink{Thesis-layout}{\XLingPaperdottedtocline{\leveltwoindent}{\leveltwowidth}{{1.1. } Thesis layout}{5}
}}\settowidth{\leveltwoindent}{{1. }\ }\settowidth{\leveltwowidth}{{1.2. }\thinspace\thinspace}\addtolength{\leveltwoindent}{.3in}{\singlespacing
\hyperlink{Core-Concepts}{\XLingPaperdottedtocline{\leveltwoindent}{\leveltwowidth}{{1.2. } Some core concepts and terms}{8}
}}\settowidth{\leveltwoindent}{{1. }\ {1.2. }\ }\settowidth{\leveltwowidth}{{1.2.1. }\thinspace\thinspace}\addtolength{\leveltwoindent}{.3in}{\singlespacing
\hyperlink{s}{\XLingPaperdottedtocline{\leveltwoindent}{\leveltwowidth}{{1.2.1. } Model of character components and make up}{9}
}}\settowidth{\levelonewidth}{2. \ }{\singlespacing
\hyperlink{c1}{\XLingPaperdottedtocline{.3in}{\levelonewidth}{2. \ Writing, text-input, and typing with keyboards}{10}
}}\settowidth{\leveltwoindent}{{2. }\ }\settowidth{\leveltwowidth}{{2.1. }\thinspace\thinspace}\addtolength{\leveltwoindent}{.3in}{\singlespacing
\hyperlink{s1.1}{\XLingPaperdottedtocline{\leveltwoindent}{\leveltwowidth}{{2.1. } Language Documentation versus Language Description}{10}
}}\settowidth{\leveltwoindent}{{2. }\ }\settowidth{\leveltwowidth}{{2.2. }\thinspace\thinspace}\addtolength{\leveltwoindent}{.3in}{\singlespacing
\hyperlink{s1.2}{\XLingPaperdottedtocline{\leveltwoindent}{\leveltwowidth}{{2.2. } The Digital Revolution}{11}
}}\settowidth{\leveltwoindent}{{2. }\ }\settowidth{\leveltwowidth}{{2.3. }\thinspace\thinspace}\addtolength{\leveltwoindent}{.3in}{\singlespacing
\hyperlink{s1.3}{\XLingPaperdottedtocline{\leveltwoindent}{\leveltwowidth}{{2.3. } The Endangered Language Movement}{13}
}}\settowidth{\leveltwoindent}{{2. }\ {2.3. }\ }\settowidth{\leveltwowidth}{{2.3.1. }\thinspace\thinspace}\addtolength{\leveltwoindent}{.3in}{\singlespacing
\hyperlink{s1.3.1}{\XLingPaperdottedtocline{\leveltwoindent}{\leveltwowidth}{{2.3.1. } Defining Language Development}{15}
}}\settowidth{\leveltwoindent}{{2. }\ }\settowidth{\leveltwowidth}{{2.4. }\thinspace\thinspace}\addtolength{\leveltwoindent}{.3in}{\singlespacing
\hyperlink{s1.4}{\XLingPaperdottedtocline{\leveltwoindent}{\leveltwowidth}{{2.4. } Writing in Society}{15}
}}\settowidth{\leveltwoindent}{{2. }\ }\settowidth{\leveltwowidth}{{2.5. }\thinspace\thinspace}\addtolength{\leveltwoindent}{.3in}{\singlespacing
\hyperlink{s.1.5}{\XLingPaperdottedtocline{\leveltwoindent}{\leveltwowidth}{{2.5. } The Role and impact of design}{15}
}}\settowidth{\leveltwoindent}{{2. }\ }\settowidth{\leveltwowidth}{{2.6. }\thinspace\thinspace}\addtolength{\leveltwoindent}{.3in}{\singlespacing
\hyperlink{s1.6}{\XLingPaperdottedtocline{\leveltwoindent}{\leveltwowidth}{{2.6. } The Role and impact of technical social systems}{15}
}}\settowidth{\leveltwoindent}{{2. }\ }\settowidth{\leveltwowidth}{{2.7. }\thinspace\thinspace}\addtolength{\leveltwoindent}{.3in}{\singlespacing
\hyperlink{s1.7}{\XLingPaperdottedtocline{\leveltwoindent}{\leveltwowidth}{{2.7. } The objectification of languages}{15}
}}\settowidth{\leveltwoindent}{{2. }\ {2.7. }\ }\settowidth{\leveltwowidth}{{2.7.1. }\thinspace\thinspace}\addtolength{\leveltwoindent}{.3in}{\singlespacing
\hyperlink{s1.7.1}{\XLingPaperdottedtocline{\leveltwoindent}{\leveltwowidth}{{2.7.1. } Objectification of the language}{15}
}}\settowidth{\leveltwoindent}{{2. }\ {2.7. }\ }\settowidth{\leveltwowidth}{{2.7.2. }\thinspace\thinspace}\addtolength{\leveltwoindent}{.3in}{\singlespacing
\hyperlink{s1.7.2}{\XLingPaperdottedtocline{\leveltwoindent}{\leveltwowidth}{{2.7.2. } Object Culture}{15}
}}\settowidth{\levelonewidth}{3. \ }{\singlespacing
\hyperlink{c2}{\XLingPaperdottedtocline{.3in}{\levelonewidth}{3. \ Methods}{16}
}}\settowidth{\leveltwoindent}{{3. }\ }\settowidth{\leveltwowidth}{{3.1. }\thinspace\thinspace}\addtolength{\leveltwoindent}{.3in}{\singlespacing
\hyperlink{s2.1}{\XLingPaperdottedtocline{\leveltwoindent}{\leveltwowidth}{{3.1. } (Methodology) Methodological considerations}{16}
}}\settowidth{\leveltwoindent}{{3. }\ {3.1. }\ }\settowidth{\leveltwowidth}{{3.1.1. }\thinspace\thinspace}\addtolength{\leveltwoindent}{.3in}{\singlespacing
\hyperlink{s2.1.1}{\XLingPaperdottedtocline{\leveltwoindent}{\leveltwowidth}{{3.1.1. } Keyboards}{16}
}}\settowidth{\leveltwoindent}{{3. }\ {3.1. }\ }\settowidth{\leveltwowidth}{{3.1.2. }\thinspace\thinspace}\addtolength{\leveltwoindent}{.3in}{\singlespacing
\hyperlink{s2.1.2}{\XLingPaperdottedtocline{\leveltwoindent}{\leveltwowidth}{{3.1.2. } Orthographies}{16}
}}\settowidth{\leveltwoindent}{{3. }\ {3.1. }\ }\settowidth{\leveltwowidth}{{3.1.3. }\thinspace\thinspace}\addtolength{\leveltwoindent}{.3in}{\singlespacing
\hyperlink{s2.1.3}{\XLingPaperdottedtocline{\leveltwoindent}{\leveltwowidth}{{3.1.3. } Typing behaviors}{16}
}}\settowidth{\leveltwoindent}{{3. }\ {3.1. }\ }\settowidth{\leveltwowidth}{{3.1.4. }\thinspace\thinspace}\addtolength{\leveltwoindent}{.3in}{\singlespacing
\hyperlink{s2.1.4}{\XLingPaperdottedtocline{\leveltwoindent}{\leveltwowidth}{{3.1.4. } Current Design Processes}{16}
}}\settowidth{\levelthreeindent}{{3. }\ {3.1. }\ {3.1.4. }\ }\settowidth{\levelthreewidth}{{3.1.4.1. }\thinspace\thinspace}\addtolength{\levelthreeindent}{.3in}{\singlespacing
\hyperlink{s2.1.4.1}{\XLingPaperdottedtocline{\levelthreeindent}{\levelthreewidth}{{3.1.4.1. } The design of orthographies and keyboards}{16}
}}\settowidth{\levelthreeindent}{{3. }\ {3.1. }\ {3.1.4. }\ }\settowidth{\levelthreewidth}{{3.1.4.2. }\thinspace\thinspace}\addtolength{\levelthreeindent}{.3in}{\singlespacing
\hyperlink{s2.1.4.2}{\XLingPaperdottedtocline{\levelthreeindent}{\levelthreewidth}{{3.1.4.2. } Good Design}{17}
}}\settowidth{\leveltwoindent}{{3. }\ }\settowidth{\leveltwowidth}{{3.2. }\thinspace\thinspace}\addtolength{\leveltwoindent}{.3in}{\singlespacing
\hyperlink{s2.2}{\XLingPaperdottedtocline{\leveltwoindent}{\leveltwowidth}{{3.2. } Orthography text samples and analyzed keyboard layouts}{17}
}}\settowidth{\levelonewidth}{4. \ }{\singlespacing
\hyperlink{c3}{\XLingPaperdottedtocline{.3in}{\levelonewidth}{4. \ (Results)The data to be explored}{18}
}}\settowidth{\leveltwoindent}{{4. }\ }\settowidth{\leveltwowidth}{{4.1. }\thinspace\thinspace}\addtolength{\leveltwoindent}{.3in}{\singlespacing
\hyperlink{s31}{\XLingPaperdottedtocline{\leveltwoindent}{\leveltwowidth}{{4.1. } impacts}{18}
}}\settowidth{\leveltwoindent}{{4. }\ {4.1. }\ }\settowidth{\leveltwowidth}{{4.1.1. }\thinspace\thinspace}\addtolength{\leveltwoindent}{.3in}{\singlespacing
\hyperlink{s311}{\XLingPaperdottedtocline{\leveltwoindent}{\leveltwowidth}{{4.1.1. } objectification of the orthography}{18}
}}\settowidth{\leveltwoindent}{{4. }\ {4.1. }\ }\settowidth{\leveltwowidth}{{4.1.2. }\thinspace\thinspace}\addtolength{\leveltwoindent}{.3in}{\singlespacing
\hyperlink{s312}{\XLingPaperdottedtocline{\leveltwoindent}{\leveltwowidth}{{4.1.2. } objectification of the keyboard, and the keyboard layout}{18}
}}\settowidth{\leveltwoindent}{{4. }\ }\settowidth{\leveltwowidth}{{4.2. }\thinspace\thinspace}\addtolength{\leveltwoindent}{.3in}{\singlespacing
\hyperlink{s32}{\XLingPaperdottedtocline{\leveltwoindent}{\leveltwowidth}{{4.2. } What should a keyboard layout enable people to do?}{18}
}}\settowidth{\leveltwoindent}{{4. }\ {4.2. }\ }\settowidth{\leveltwowidth}{{4.2.1. }\thinspace\thinspace}\addtolength{\leveltwoindent}{.3in}{\singlespacing
\hyperlink{s321}{\XLingPaperdottedtocline{\leveltwoindent}{\leveltwowidth}{{4.2.1. } write in their language, in their script}{18}
}}\settowidth{\leveltwoindent}{{4. }\ {4.2. }\ }\settowidth{\leveltwowidth}{{4.2.2. }\thinspace\thinspace}\addtolength{\leveltwoindent}{.3in}{\singlespacing
\hyperlink{s322}{\XLingPaperdottedtocline{\leveltwoindent}{\leveltwowidth}{{4.2.2. } Control the computer}{18}
}}\settowidth{\leveltwoindent}{{4. }\ }\settowidth{\leveltwowidth}{{4.3. }\thinspace\thinspace}\addtolength{\leveltwoindent}{.3in}{\singlespacing
\hyperlink{s33}{\XLingPaperdottedtocline{\leveltwoindent}{\leveltwowidth}{{4.3. } What components does the framework need to contain?}{19}
}}\settowidth{\levelonewidth}{5. \ }{\singlespacing
\hyperlink{c4}{\XLingPaperdottedtocline{.3in}{\levelonewidth}{5. \ Methodology}{20}
}}\settowidth{\leveltwoindent}{{5. }\ }\settowidth{\leveltwowidth}{{5.1. }\thinspace\thinspace}\addtolength{\leveltwoindent}{.3in}{\singlespacing
\hyperlink{s41}{\XLingPaperdottedtocline{\leveltwoindent}{\leveltwowidth}{{5.1. } UX Analysis}{20}
}}\settowidth{\leveltwoindent}{{5. }\ }\settowidth{\leveltwowidth}{{5.2. }\thinspace\thinspace}\addtolength{\leveltwoindent}{.3in}{\singlespacing
\hyperlink{s42}{\XLingPaperdottedtocline{\leveltwoindent}{\leveltwowidth}{{5.2. } Methods in UX analysis}{20}
}}\settowidth{\leveltwoindent}{{5. }\ {5.2. }\ }\settowidth{\leveltwowidth}{{5.2.1. }\thinspace\thinspace}\addtolength{\leveltwoindent}{.3in}{\singlespacing
\hyperlink{s421}{\XLingPaperdottedtocline{\leveltwoindent}{\leveltwowidth}{{5.2.1. } Specific methods related the acquisition of my data}{20}
}}\settowidth{\levelthreeindent}{{5. }\ {5.2. }\ {5.2.1. }\ }\settowidth{\levelthreewidth}{{5.2.1.1. }\thinspace\thinspace}\addtolength{\levelthreeindent}{.3in}{\singlespacing
\hyperlink{s4211}{\XLingPaperdottedtocline{\levelthreeindent}{\levelthreewidth}{{5.2.1.1. } Keystroke Counting}{20}
}}\settowidth{\levelthreeindent}{{5. }\ {5.2. }\ {5.2.1. }\ }\settowidth{\levelthreewidth}{{5.2.1.2. }\thinspace\thinspace}\addtolength{\levelthreeindent}{.3in}{\singlespacing
\hyperlink{s4212}{\XLingPaperdottedtocline{\levelthreeindent}{\levelthreewidth}{{5.2.1.2. } Survey Data}{20}
}}\settowidth{\leveltwoindent}{{5. }\ }\settowidth{\leveltwowidth}{{5.3. }\thinspace\thinspace}\addtolength{\leveltwoindent}{.3in}{\singlespacing
\hyperlink{s44}{\XLingPaperdottedtocline{\leveltwoindent}{\leveltwowidth}{{5.3. } The Role of linguistic knowledge in UX}{20}
}}\settowidth{\levelonewidth}{6. \ }{\singlespacing
\hyperlink{c5}{\XLingPaperdottedtocline{.3in}{\levelonewidth}{6. \ The results of several languages}{21}
}}\settowidth{\leveltwoindent}{{6. }\ }\settowidth{\leveltwowidth}{{6.1. }\thinspace\thinspace}\addtolength{\leveltwoindent}{.3in}{\singlespacing
\hyperlink{s51}{\XLingPaperdottedtocline{\leveltwoindent}{\leveltwowidth}{{6.1. } Use Case \#1 Me'phaa}{21}
}}\settowidth{\leveltwoindent}{{6. }\ {6.1. }\ }\settowidth{\leveltwowidth}{{6.1.1. }\thinspace\thinspace}\addtolength{\leveltwoindent}{.3in}{\singlespacing
\hyperlink{s511}{\XLingPaperdottedtocline{\leveltwoindent}{\leveltwowidth}{{6.1.1. } Phonology}{21}
}}\settowidth{\leveltwoindent}{{6. }\ {6.1. }\ }\settowidth{\leveltwowidth}{{6.1.2. }\thinspace\thinspace}\addtolength{\leveltwoindent}{.3in}{\singlespacing
\hyperlink{s512}{\XLingPaperdottedtocline{\leveltwoindent}{\leveltwowidth}{{6.1.2. } Orthography}{21}
}}\settowidth{\leveltwoindent}{{6. }\ {6.1. }\ }\settowidth{\leveltwowidth}{{6.1.3. }\thinspace\thinspace}\addtolength{\leveltwoindent}{.3in}{\singlespacing
\hyperlink{s513}{\XLingPaperdottedtocline{\leveltwoindent}{\leveltwowidth}{{6.1.3. } Keyboard Layout}{21}
}}\settowidth{\leveltwoindent}{{6. }\ {6.1. }\ }\settowidth{\leveltwowidth}{{6.1.4. }\thinspace\thinspace}\addtolength{\leveltwoindent}{.3in}{\singlespacing
\hyperlink{s514}{\XLingPaperdottedtocline{\leveltwoindent}{\leveltwowidth}{{6.1.4. } Social Use setting of typing in the language}{21}
}}\settowidth{\leveltwoindent}{{6. }\ {6.1. }\ }\settowidth{\leveltwowidth}{{6.1.5. }\thinspace\thinspace}\addtolength{\leveltwoindent}{.3in}{\singlespacing
\hyperlink{s515}{\XLingPaperdottedtocline{\leveltwoindent}{\leveltwowidth}{{6.1.5. } Statistics from Text Analysis}{21}
}}\settowidth{\leveltwoindent}{{6. }\ {6.1. }\ }\settowidth{\leveltwowidth}{{6.1.6. }\thinspace\thinspace}\addtolength{\leveltwoindent}{.3in}{\singlespacing
\hyperlink{s516}{\XLingPaperdottedtocline{\leveltwoindent}{\leveltwowidth}{{6.1.6. } Observations}{21}
}}\settowidth{\leveltwoindent}{{6. }\ }\settowidth{\leveltwowidth}{{6.2. }\thinspace\thinspace}\addtolength{\leveltwoindent}{.3in}{\singlespacing
\hyperlink{s52}{\XLingPaperdottedtocline{\leveltwoindent}{\leveltwowidth}{{6.2. } Use Case \#2 Chinantec}{21}
}}\settowidth{\leveltwoindent}{{6. }\ }\settowidth{\leveltwowidth}{{6.3. }\thinspace\thinspace}\addtolength{\leveltwoindent}{.3in}{\singlespacing
\hyperlink{s53}{\XLingPaperdottedtocline{\leveltwoindent}{\leveltwowidth}{{6.3. } Use Case \#3 Spanish}{21}
}}\settowidth{\leveltwoindent}{{6. }\ }\settowidth{\leveltwowidth}{{6.4. }\thinspace\thinspace}\addtolength{\leveltwoindent}{.3in}{\singlespacing
\hyperlink{s54}{\XLingPaperdottedtocline{\leveltwoindent}{\leveltwowidth}{{6.4. } Use Case \#4 English}{21}
}}\settowidth{\leveltwoindent}{{6. }\ }\settowidth{\leveltwowidth}{{6.5. }\thinspace\thinspace}\addtolength{\leveltwoindent}{.3in}{\singlespacing
\hyperlink{s55}{\XLingPaperdottedtocline{\leveltwoindent}{\leveltwowidth}{{6.5. } Use Case \#5 ??? - from Africa}{22}
}}\settowidth{\levelonewidth}{7. \ }{\singlespacing
\hyperlink{c6}{\XLingPaperdottedtocline{.3in}{\levelonewidth}{7. \ What we can observe from these Use Cases and layouts}{23}
}}{\singlespacing
\hyperlink{rXLingPapAppendiciesPage}{\XLingPaperdottedtocline{0pt}{0pt}{APPENDICES}{24}
}}{\singlespacing
\hyperlink{rXLingPapGlossary1}{\XLingPaperdottedtocline{0pt}{0pt}{Appendex I: Glossary of technical concepts and terms}{47}
}}\clearpage
\thispagestyle{frontmatterfirstpage}{\needspace{3\baselineskip}
\vspace*{.65in}\noindent
\raisebox{\baselineskip}[0pt]{\pdfbookmark[1]{List of tables}{rXLingPapPreface1}}\raisebox{\baselineskip}[0pt]{\protect\hypertarget{rXLingPapPreface1}{}}\noindent
{\MakeUppercase{{\protect\centering
List of tables\\}}}\markboth{List of tables}{List of tables}
\XLingPaperaddtocontents{rXLingPapPreface1}}\penalty10000\par{}
\vspace{10.8pt}\noindent{}{Table }\hfill{}Page\par{}
\clearpage
\thispagestyle{frontmatterfirstpage}{\needspace{3\baselineskip}
\vspace*{.65in}\noindent
\raisebox{\baselineskip}[0pt]{\pdfbookmark[1]{List of figures}{rXLingPapPreface2}}\raisebox{\baselineskip}[0pt]{\protect\hypertarget{rXLingPapPreface2}{}}\noindent
{\MakeUppercase{{\protect\centering
List of figures\\}}}\markboth{List of figures}{List of figures}
\XLingPaperaddtocontents{rXLingPapPreface2}}\penalty10000\par{}
\vspace{10.8pt}\noindent{}{Figure }\hfill{}Page\par{}
{\singlespacing
\hyperlink{f-issues-mindmap}{\XLingPaperdottedtocline{0pt}{\XLingPapersingledigitlistofwidth{}}{1.  Various issues affecting the development of social literacy in digital mediums for minority language speakers}{7}
}}{\singlespacing
\hyperlink{f-character-make-up}{\XLingPaperdottedtocline{0pt}{\XLingPapersingledigitlistofwidth{}}{2.  Characters}{9}
}}\clearpage
\thispagestyle{frontmatterfirstpage}{\needspace{3\baselineskip}
\vspace*{.65in}\noindent
\raisebox{\baselineskip}[0pt]{\pdfbookmark[1]{List of abbreviations}{rXLingPapPreface3}}\raisebox{\baselineskip}[0pt]{\protect\hypertarget{rXLingPapPreface3}{}}\noindent
{\MakeUppercase{{\protect\centering
List of abbreviations\\}}}\markboth{List of abbreviations}{List of abbreviations}
\XLingPaperaddtocontents{rXLingPapPreface3}}\penalty10000\par{}
\vspace{10.8pt}\clearpage
\thispagestyle{frontmatterfirstpage}{\needspace{3\baselineskip}
\vspace*{.65in}\noindent
\raisebox{\baselineskip}[0pt]{\pdfbookmark[1]{Acknowledgements}{rXLingPapAcknowledgements}}\raisebox{\baselineskip}[0pt]{\protect\hypertarget{rXLingPapAcknowledgements}{}}\noindent
{\MakeUppercase{{\protect\centering
Acknowledgements\\}}}\markboth{Acknowledgements}{Acknowledgements}
\XLingPaperaddtocontents{rXLingPapAcknowledgements}}\penalty10000\par{}
\vspace{10.8pt}\indent I would like to thank Bob Hallissy of SIL International, NRSI for help understanding Unicode and converting non-Unicode texts to Unicode; Mark L. Weathers for his extensive knowledge of the use of technology in writing Meꞌphaa; Steve Marlett for inviting me to work with him on documenting the Meꞌphaa genus; Wilfrido Flores for giving me the opportunity to serve him in the creation of a Chinantec keyboard; John Brownie, the creator of Ukelele a program which makes the designing of keyboards layouts on OS X a really easy thing to do; Kevin Cline for explaining to me how MSKLC works and for creating the MSKLC keyboards for Meꞌphaa; Tim Sissel of SIL Mexico Branch for is assistance with the history of keyboard design in SIL's language program involvement in Mexico; John Gibbens for assistance in obtaining Mongolian texts and keyboard layouts for analysis; Coleen Starwalt and Heidi Rosendall for assistance in obtaining Mongolian texts and keyboard layouts for analysis; Kirby O’Brien for his involvement in refining some of the graphics used to demonstrate some of the ideas of this thesis; Becky Paterson for proof reading and helping a visual processor like me reduce my thoughts to keystrokes. SIL International Americas Area for sponsoring my involvement in the Meꞌphaa genus Language Documentation project. I take full responsibility for all errors.\par{}\indent This thesis was proudly typeset with \href{http://www.xlingpaper.org/}{\textcolor[rgb]{0,0,0}{XLingPaper}}.\par{}\clearpage
\thispagestyle{frontmatterfirstpage}{\needspace{3\baselineskip}
\vspace*{.65in}\noindent
\raisebox{\baselineskip}[0pt]{\pdfbookmark[1]{Abstract}{rXLingPapAbstract}}\raisebox{\baselineskip}[0pt]{\protect\hypertarget{rXLingPapAbstract}{}}\noindent
{\MakeUppercase{{\protect\centering
Abstract\\}}}\markboth{Abstract}{Abstract}
\XLingPaperaddtocontents{rXLingPapAbstract}}\penalty10000\par{}
\vspace{10.8pt}\clearpage
\pagestyle{body}\pagenumbering{arabic}\thispagestyle{bodyfirstpage}\markboth{Introduction}{Introduction}
\XLingPaperaddtocontents{c}{\vspace*{.65in}\noindent
\fontsize{14}{16.8}\selectfont \textbf{{\centering
CHAPTER \raisebox{\baselineskip}[0pt]{\protect\hypertarget{c}{}}\raisebox{\baselineskip}[0pt]{\pdfbookmark[1]{1 Introduction}{c}}1\\}}}\par{}
{\noindent
\fontsize{14}{16.8}\selectfont \textbf{{\centering
Introduction\\}}}\par{}
\vspace{16pt}\indent Keyboard layout design affects language vitality. Socio-technical systems are increasingly important in today's communication ecology (Whitworth \& Ahmad 2013). Language development projects and language planing programs need a way to integrate linguistic knowledge, information, and transmission practices into socio-technical systems if the languages used in these systems are going to be the mother tongue languages of minority language speakers. With the current rate of technological adaption it is more than feasible that technical systems (such as social media and the mobile devices used to access these systems) will become more relevant than the traditional, printed, literacy reading primer (Blench 2012: 15). This requires addressing the design tension between requirements for minority language users and the Human Computer Interaction (HCI) requirements of computing devices. The academic linguistic community often attempts to address these tensions at the orthography “design” level (Cooper 2005: 160, Jany 2010b: 235-6). However these “solutions” often revolve around removing diacritic marks from Roman script orthographies (Boerger 2007: 134) and do not address the marking of tone in languages, such as Chinantec (Foris 2000) and some African languages (Roberts 2011), where there is a significant need to mark tone. Such solutions also do not affect key frequency issues, or diacritic marks in Ajami and Indic scripts. This project focuses on the arrangement of keys on the keyboard, or keyboard layout (KL); proposing that KL’s are the cornerstone to truly adapting the digital content creation process to the needs of minority language users. In the context of minority language text input design specifications and considerations, there has been relatively little published, either for the publishing industry, linguists, or for technologists (designers and programmers). The one exception is an unfinished book released in draft form by SIL’s foundry NRSI (Lyons 2001). In contrast to the relatively sparse literature specifically supporting and covering minority language text input, QWERTY keyboard interactions, primarily dealing with English, are well studied (MacKenzie 2002, 2007, 2013, MacKenzie \& Tanaka-Ishii 2007). This current study takes current practice in the HCI literature and applies it to several minority language use cases, focusing on languages which use diacritics, often as a device used to explicitly mark tone in their orthographies.\par{}\indent In communicative environments where there is the option to use more than one language, choice of language is based in both social and physical environments. Orthography design decisions are often perceived to have an effect on the mechanics of language expression in digital mediums. However, strictly speaking it is only the text input method not the orthography which plays a role in the mechanics of creating new entextualized content in digital mediums. Emotional responses to physical elements of a language such as the design of orthography, the computer operating system, and of the text input method bear upon the language user. In the disciplines of language documentation and language description, text input methods may initially be developed with the needs of the researcher in mind rather than the needs of a native speaker who uses the language in everyday interactions. These existing keyboard layouts that support specific languages,which are created by researchers, are rarely used by the broader minority language community, and the efficacy of these keyboard layouts is limited to linguistic analysis or researcher convenience. Linguists often bring linguistic knowledge and some of their own expectations as ‘experienced’ computer users to the keyboard design process. They may not realize that requiring a typist to negotiate a keyboard layout to access a given character (often a non-ASCII character) can have an impact on language-use choice, the development path of an orthography, or adherence to an approved orthography. User-centric keyboard layout design for minority language community writers/typists should be an integral part of a language development project in the twenty-first century. These considerations bring us to the following question: At what point in the design process should linguistic information be considered and applied, as opposed to other design criteria, so that maximal language usage is encouraged and made possible? This study offers a framework for the linguist or language development worker to address crucial issues of keyboard layout design.\par{}\indent There are four reasons that the mechanical process of writing and the process of typing in digital contexts (text input) is of interest to those who study languages: they are an expression of thought, a means of communication, and a reflection of brain processes. First, keyboarding is an expression of language and reveals some very unique ways that the human body expresses communicative thought. For instance, consider the ability to type ‘LOL’ without actually laughing or thinking “laugh out loud”. These typing gestures can connect with our thoughts without activating the vocal or aural mechanisms which are often employed in the encoding and decoding of communicative thought. The study of the mechanics of writing is not new. European Renaissance writers were discussing hand writing in relationship to personality, as early as the sixteenth century (Baldi 1622). More recent works focus on: the relationship between handwriting and brain processing (Askov, Otto \& Askov 1970, Peck, Askov \& Fairchild 1980), motor control (van Galen 1991), and the developmental and pedagogical change insinuated by moving from handwriting to typing as the mechanical bases of the expression of textual compositions (Christensen 2004, Stevenson \& Just 2012). Second, the language teacher (including second language instructor) is interested in language use in all mediums; computer-mediated communication, and oral communication. Chapelle (2003) and Jones \& Plass (2002) differ in how they conceptualize the integration of technology use in the language learning process. However, regardless of the theoretical approach, typing and keyboard input is an acknowledged component of the Computer Assisted Language Learning (CALL) environment (Strik 2012: 9) if nothing else but to facilitate other more salient aspects of learning theory activities. More specifically though Lally (2000) and Sturm (2006) argue that keyboarding and typing does effect the way that language learners recall the use of diacritics on words. The third reason that typing (text input) is of interest to those who study language is that the psychologist and linguist are interested in how the brain processes language through the process of writing, which includes typing (text input). This thesis will touch on various aspects of these three points as it proposes a framework for keyboard layout design. The fourth reason that is text input is important to those who study language is that text input is important in the language development movement. That is, as more and more minority language using communities approach the task of language development they often reach out to those who study languages (linguists) for help. As Lally (2000) and Sturm (2006) argue that keyboarding affect the way that language learners remember characters, it seems logical then that for learners of languages, even if they are native speakers, or heritage learners would be subject to the same impacts of typing on the way they learn the written form of the language they are using. This is an important point which needs to be worked into language development practices by those involved in language development activities like orthography development which in some sense can be a sub-component of text input development or writing development. In particular, diacritics and their use in orthographies become important since it is estimated that between 60-70 percent of languages are tonal (Yip 2002: 1) and diacritics are the primary way orthographies indicate tone.\par{}\indent Keyboard layout design is intrinsically interdisciplinary. To create a tool for language use which not only works but is embraced by a group of users requires an understanding of linguistic knowledge, script knowledge, and digital technology knowledge germane to the language entextulization challenge. For a new keyboard layout (analyzed as an object) to be embraced by a user group requires a successful application of principles from economics, anthropology, and design, especially user experience design. That is, people must be able to access the object, want to use the object, and finally choose to use the object.\par{}{\needspace{3\baselineskip}
\vspace{15pt}\noindent
\fontsize{13}{15.6}\selectfont \textbf{{\noindent
\raisebox{\baselineskip}[0pt]{\pdfbookmark[2]{{1.1 } Thesis layout}{Thesis-layout}}\raisebox{\baselineskip}[0pt]{\protect\hypertarget{Thesis-layout}{}}{1.1 }Thesis layout}}\markright{Thesis layout}
\XLingPaperaddtocontents{Thesis-layout}}\par{}
\penalty10000\vspace{10pt}\penalty10000\indent Chapter one of the thesis provides a brief introduction to the topic of keyboard layouts. It presents the relevance of the study of text input to linguistics. It also provides an overview of the various chapters in the thesis and a discussion of key concepts and terms used throughout the thesis.\par{}\indent Chapter two of this thesis takes the reader through the relevance of writing to the disciplines of linguistics and language development. It is often within this context that new keyboard layouts are created for monitory languages. The first section discusses entextulization and the process often followed in developing writing for the purposes of linguistic research, language documentation and language development. These settings are not without conflicting views surrounding writing as a part of language development. Just as writing is affected by various social practices and communal attitudes towards writing, so also the process of typing (text input) is affected by similar social constraints. That is, the need for writing, and therefore also the need for text input, is not felt ubiquitously.\par{}\indent Included in chapter two is an introduction to writing and discussion of the current literature relevant to human computer interaction (HCI) and keyboard interaction analysis. Academically, user experience design falls under the broader discipline of computer science, therefore much of the literature discussing text input (even in minority languages) does not occur in the linguistics or language documentation literature. Current literature concerning keyboard layout design, while not solely based on English language text input, is predominantly based on English language research. Furthermore, this research is rarely cited and apparently un-accessed by language development staff in the production of keyboard layouts (p.c. with various keyboard layout designers). For these language development staff a far more pressing goal is the correct typesetting of professional documents, therefore the keyboard layout becomes a way to limit (or quality control) data input options for text processing systems\protect\footnote[1]{{\leftskip0pt\parindent1em\raisebox{\baselineskip}[0pt]{\protect\hypertarget{n-NeedsALabel-.xlingpaper.1..styledPaper.1..lingPaper.1..chapter.1..section1.1..p.3..endnote.1.}{}} The creation of keyboard layouts and text input systems is sometimes delegated to publishers (and typesetters and their foundries). These stakeholders in the publishing process are very interested in consistent encoding of texts. As an example some packages of LaTeX require the special declaration of combining glyphs to form characters and can not accept strait Unicode characters (Goossens, Rahtz \& Mittelbach 1997: 264-5). This more restricted approach to text input can be seen as a challenge for self publishers, who prefer a more straightforward approach to entextualization.}}. It is the goal of this thesis to integrate HCI and language development literatures so that the language development professional has a resource which references both literatures and provides that person with a framework upon which to design future keyboard layouts. Figure 1 is a visualization of the various topics discussed in this thesis and their inter-relatedness. It attempts to layout the topical landscape on two clines: the community internal - external cline (right and left sides), and the issues affecting the desire and capability of a community to engage in the act of writing (top and bottom). Connecting the various topics are several classes of lines which generally show some sort of association, though the association is not always the same in every language’s situation. Heavier lines generally show more relatedness, while dotted lines show an amorphous relationship. Arrows generally show direction of impact when a directionality is determinable.\par{}\indent Chapters three and four of this thesis present a comparative study of the alleged typing experience in fifteen languages. Thirteen of these languages use the Latin script (also know as the Roman script), and two of these languages use the cyrillic script. The Latin script based languages contain a variety of diacritics, and diacritic use frequencies. The Book of James is used as a corpus to derive keystrokes. These keystrokes and their frequencies are then compared and used to make suggestions for keyboard layout designs.\par{}\indent Chapter five highlights some outstanding issues in keyboard layout design in terms of theory, technology, and practical application of language related knowledge to the keyboard layout design process.\par{}\indent Back matter: It is hoped that the reader finds the interdisciplinary bibliography useful. It covers the topics user experience design, orthography design, keyboard layouts, and the sociolinguistics and sociology of writing. Following the bibliography is an appendix with a short glossary of technical terms. A second appendix with a list of the technical standards referenced in this thesis. In a third appendix, for the sake of completeness and for the benefit future researchers the analyzed texts are included in their entirety.
\par{}\vspace{12pt plus 2pt minus 1pt}{\protect\raggedright\leavevmode
\vspace*{0pt}{\XeTeXpdffile "../images/f-issues-mindmap-ori.pdf" scaled 750}\\[0pt]\protect\hypertarget{f-issues-mindmap}{}\XLingPaperaddtocontents{f-issues-mindmap}{\singlespacing
{Figure }{1.}{ Various issues affecting the development of social literacy in digital mediums for minority language speakers\\}}}\vspace{12pt plus 2pt minus 1pt}{\needspace{3\baselineskip}
\vspace{15pt}\noindent
\fontsize{13}{15.6}\selectfont \textbf{{\noindent
\raisebox{\baselineskip}[0pt]{\pdfbookmark[2]{{1.2 } Some core concepts and terms}{Core-Concepts}}\raisebox{\baselineskip}[0pt]{\protect\hypertarget{Core-Concepts}{}}{1.2 }Some core concepts and terms}}\markright{Some core concepts and terms}
\XLingPaperaddtocontents{Core-Concepts}}\par{}
\penalty10000\vspace{10pt}\penalty10000\indent This section provides a cursory discussion of key concepts and terms used throughout this thesis. Some of these terms are also available in the glossary contained in the back matter. It is acknowledged that in each of the respective fields, various authors use the same term with various connotations of scope. Therefore for clarity it is necessary to address what is meant in this thesis by these terms, and it is hoped that addressing these key terms earlier rather than later in the thesis will provide additional clarity to the reader. Many of these terms are not new and exist in other, non-linguistics literatures. The goal in presenting these ‘models’ is not to articulate or present a comprehensive taxonomy of knowledge in the fields represented. Rather the purpose is to give the reader a brief overview on the issue as this thesis will reference concepts and terms from various academic disciplines and areas of practice. However, some topics in this section will be discussed in more detail than others.\par{}{\needspace{3\baselineskip}
\vspace{10pt}\noindent
\fontsize{13}{15.6}\selectfont \textit{{\noindent
\raisebox{\baselineskip}[0pt]{\pdfbookmark[3]{{1.2.1 } Model of character components and make up}{s}}\raisebox{\baselineskip}[0pt]{\protect\hypertarget{s}{}}{1.2.1 }Model of character components and make up}}\markright{Model of character components and make up}
\XLingPaperaddtocontents{s}}\par{}
\penalty10000\vspace{10pt}\penalty10000\vspace{12pt plus 2pt minus 1pt}{\protect\centering \leavevmode
\vspace*{0pt}{\XeTeXpdffile "../images/f-character-make-up.pdf" scaled 750}\\[0pt]\protect\hypertarget{f-character-make-up}{}\XLingPaperaddtocontents{f-character-make-up}{\singlespacing
{Figure }{2.}{ Characters\\}}}\vspace{12pt plus 2pt minus 1pt}\clearpage
\thispagestyle{bodyfirstpage}\markboth{Writing, text-input, and typing with keyboards}{Writing, text-input, and typing with keyboards}
\XLingPaperaddtocontents{c1}{\vspace*{.65in}\noindent
\fontsize{14}{16.8}\selectfont \textbf{{\centering
CHAPTER \raisebox{\baselineskip}[0pt]{\protect\hypertarget{c1}{}}\raisebox{\baselineskip}[0pt]{\pdfbookmark[1]{2 Writing, text-input, and typing with keyboards}{c1}}2\\}}}\par{}
{\noindent
\fontsize{14}{16.8}\selectfont \textbf{{\centering
Writing, text-input, and typing with keyboards\\}}}\par{}
\vspace{16pt}\indent Desire, ability and opportunity function together to allow minority language writers to produce written materials in their languages. Keyboards and text input are the gateway to creating these text in digital mediums. Where any one of these three factors fail to exist regardless of the digital/non-digital context text output will be affected. That said, each of these factors can be affected by the technology, through the user's interaction with the technology. Technology (either digital devices or an orthography) in and of itself is not the saviour of an endangered or minority language, though it can be the platform on which many new conversations are conducted. The reason for this is that the problems of text production are fundamentally sociological, not technical. The technical aspect surfaces as a challenge when certain sociological impacts are not achieved. The sociological task (tasks when completed result in impacts) most relevant to this thesis is the speed of communication. However, in the mind of the user it is likely the ease of communication in a written form. For the typing experience to be successful by any calculation, language users must be trying to communicate via keyboards and text input. Inherently this infers a social attitude about writing. It is this attitude which is the ultimate medium of keyboard layout designer’s craft.\par{}{\needspace{3\baselineskip}
\vspace{15pt}\noindent
\fontsize{13}{15.6}\selectfont \textbf{{\noindent
\raisebox{\baselineskip}[0pt]{\pdfbookmark[2]{{2.1 } Language Documentation versus Language Description}{s1.1}}\raisebox{\baselineskip}[0pt]{\protect\hypertarget{s1.1}{}}{2.1 }Language Documentation versus Language Description}}\markright{Language Documentation versus Language Description}
\XLingPaperaddtocontents{s1.1}}\par{}
\penalty10000\vspace{10pt}\penalty10000\indent Since the early 1990’s language documentation has emerged as its own discipline (Furbee 2010, Himmelmann 1998, Woodbury 2003) growing out of the field of linguistics. One of language documentation’s distinctives is the collection of original language use performances (Nathan 2010) in digitally archivable formats (Bird \& Simons 2003). Within linguistics the focus on primary data is a shift in paradigm (Thieberger \& Musgrave 2007: 27-9) as much as it is in methods (Bergqvist 2012: 24). Language Documentation has focused on creating lasting and multi-purpose language artifacts, where as linguistics, focusing on description, has traditionally sought to identify the patterns occurring within and around language use. Along the way, and facilitating the split between language documentation and linguistics, the field of linguistics has encountered two other notable movements: the digital revolution, and the endangered language movement. These movements have changed the the focal evidence of linguistic argumentation from being an evidence based science argued from antidotal observations by linguists worried about the observers paradox (Labov 1966, 1972, 2006), and descriptions of languages based on written forms of observed linguistic performance (for example hand transcribed Swadesh lists (Swadesh 1971: 283), to a science driven by data, rich with reviewable examples of performance (Coleman 2011, Schroeter \& Thieberger 2011, Thieberger 2009) gathered collaboratively by speakers and researchers (Dwyer 2006: 54-6, 2010, Kuhlmann 1992: esp. 277-278, Leonard \& Haynes 2010, Penfield, et al. 2008).\par{}{\needspace{3\baselineskip}
\vspace{15pt}\noindent
\fontsize{13}{15.6}\selectfont \textbf{{\noindent
\raisebox{\baselineskip}[0pt]{\pdfbookmark[2]{{2.2 } The Digital Revolution}{s1.2}}\raisebox{\baselineskip}[0pt]{\protect\hypertarget{s1.2}{}}{2.2 }The Digital Revolution}}\markright{The Digital Revolution}
\XLingPaperaddtocontents{s1.2}}\par{}
\penalty10000\vspace{10pt}\penalty10000\indent The first of these two movements is the Digital Revolution. The advent of socially embraced digital communication has affected the behavior of both the observed (Kiesler, Siegel \& McGuire 1984) and the observer (Crasborn 2010); the speaker and the listener (Seltzer, Prososki, Ziegler \& Pollak 2012); the writer (Porter 2003) and the reader (Fortunati \& Vincent 2014, Liu 2005, Mangen, Walgermo \& Brønnick 2013). Digital devices are reshaping the communicative context in which ‘language’ exists. Handheld radios are replacing surrogate speech forms in Chinantec \textsquarebracketleft{}cso\textsquarebracketright{} society (Wilfredo Flores, pc.; Mark Sicoli in segment 23:00-23:17 in D. Duncan 2013). Research in L1/L2 and L2/L1 transference, and the role of orthographies in the production of sounds suggests that devices with text based dependencies for operation stand to have the potential to expedite the reshaping of sounds in a minority language via the graphical similarity between a minority language orthography and the orthography of a language of wider communication (Detey \& Nespoulous 2008, Major 2008: 69, Perre, Pattamadilok, Montant \& Ziegler 2009, Simon, Chambless \& Kickhöfel Alves 2010, Vendelin \& Peperkamp 2006)\protect\footnote[1]{{\leftskip0pt\parindent1em\raisebox{\baselineskip}[0pt]{\protect\hypertarget{n-NeedsALabel-.xlingpaper.1..styledPaper.1..lingPaper.1..chapter.2..section1.2..p.1..endnote.1.}{}} These claims are not universally accepted. Inconclusive results are presented by Pytlyk (2007, 2011); and Pattamadilok et al. (2011: 121) while arguing for the orthographic influence on phonology point out: “… that whether orthographic knowledge affects the core mechanisms of speech processing (e.g., lexical access) or some more peripheral processes (e.g., explicit segmentation or decision/comparison) seems to depend strongly on the choice of the tasks that researchers use to probe speech processing.” For the purpose of this paper, I take this to mean that there are likely a variety of factors affecting the orthography-pronunciation relationship; of which orthography in the digital device is one.}}. The change of language use (including loss of historically spoken minority languages) is not the only impact digital devices are having on minority languages. In some contexts minority language speakers are either adapting language use habits to incorporate the use of digital devices (Lexander 2011) or adapting their language related products (orthographies) so that it can more readily be used on existing devices (Jany 2010b: 235). Digital tools not only allow for new methods of language analysis using large multimedia corpora (Crasborn, Hulsbosch, Lampen \& Sloetjes 2014), but also enable people to communicate across time and space in new ways (Brinckwirth 2012, Elia 2006, Maslamani 2013). Computer and electronic device meditated communication is a reality in current language use - both oral and written. To the 21st Century linguist this means not just studying language in its non-digital contexts, but also in its digital contexts. With the introduction of the mobile digital context, language users no-longer have a choice between the two modalities of oral v.s written, rather there is a complex array of options available to most people which cover a plethora of communicative devices and multi-modal/multi-medium scenarios. For example, interlocutor ‘A’ may get a short email message on his computer from interlocutor ‘B’ and reply via the ‘Facebook Chat’ app via his mobile device and carry on several exchanges with interlocutor ‘B’ before walking into interlocutor ‘B’s’ office and continuing the conversation orally. All the while each segment of the conversation is constrained by the medium it occurs within. Computers have larger screens and keyboards, smart phones have smaller screens and smaller keyboards but also have oral to text features packaged with their Operating Systems, while oral face-to-face communication usually carries with it a host of visual cues and ‘reduced’ speech forms.\par{}{\needspace{3\baselineskip}
\vspace{15pt}\noindent
\fontsize{13}{15.6}\selectfont \textbf{{\noindent
\raisebox{\baselineskip}[0pt]{\pdfbookmark[2]{{2.3 } The Endangered Language Movement}{s1.3}}\raisebox{\baselineskip}[0pt]{\protect\hypertarget{s1.3}{}}{2.3 }The Endangered Language Movement}}\markright{The Endangered Language Movement}
\XLingPaperaddtocontents{s1.3}}\par{}
\penalty10000\vspace{10pt}\penalty10000\indent The second of these two movements is the endangered language movement. The endangered language movement can be broken down into two main tenants: document and develop. Krauss is credited with sounding the cry which started this movement (Hale, et al. 1992:9). Sounding a cry that linguists have not only a responsibility to study and document these disappearing languages but also to assist their speakers in the task of developing their languages. He says:\par{}\XLingPaperblockquote{.25in}{{\singlespacing
\vspace{-1.3\baselineskip}We should not only be documenting these languages, but also working educationally, culturally, and politically to increase their chances of survival. This means working with members of the relevant communities to help produce pedagogical materials and literature and to promote language development in the necessary domains, including television.\par{}}}{\baselineskip}{\baselineskip}\noindent In the context of the endangered language movement, it is important to distinguish language development from language documentation. Unlike language documentation, language development is not a new concept; being defined as early as 1968 by Ferguson (1968). The distinction between language documentation and language development is pressed by Simons (2011), nineteen years after Krauss\protect\footnote[2]{{\leftskip0pt\parindent1em\raisebox{\baselineskip}[0pt]{\protect\hypertarget{n-NeedsALabel-.xlingpaper.1..styledPaper.1..lingPaper.1..chapter.2..section1.3..pc.1..endnote.1.}{}} Although Ferguson (1968) also does layout much of the same definition for language development.}}. Simons defines language development as:\par{}\XLingPaperblockquote{.25in}{{\singlespacing
\vspace{-1.3\baselineskip}… activities undertaken for the purpose of developing new functions for its language or for restoring lost functions.\par{}}}{\baselineskip}{\baselineskip}\noindent There are two pertinent remarks when considering Simons’ definition. The first relates to the first movement mentioned previously - The Digital Revolution. That is, for many languages ‘new domains’ would include making the language viable in digital contexts, be it written, or oral, or oral with visual support (like YouTube and more generally all kinds of video). The second remark is that the long time delay in formally defining ‘language development’ does not mean that development and development-like activities were not undertaken prior to Krauss’ call to action or in the interim between Krauss and Simons’ formalized definition. Blench (2012: 13) generalizes the language development pattern in a Nigerian context for the past century saying:\par{}\XLingPaperblockquote{.25in}{{\singlespacing
\vspace{-1.3\baselineskip}A language was first analysed linguistically, a draft orthography was developed, primers to teach the language were printed, and as literacy initiatives were undertaken, Bible translations were very often begun. Wherever literacy took off, in major languages such as Hausa and Yoruba, this would ‘leak’ into the secular sphere. Books, newspapers and advertising would pick up on the possibility of targeting specific ethnic audiences.\par{}}}{\baselineskip}{\baselineskip}\noindent Krauss’ call has not gone un-heeded, and in contrast to the characterization of events over the last century provided by Blench, the journal Language Documentation \& Conservation is rife with more recent examples, perspectives, and use cases where linguists have engaged with communities to help “develop” their languages (Amery 2009, Otsuka \& Wong 2007, Yamada 2007). Often these use-cases use the term ‘language revitalization’ to describe their language development type activities. Many ethnolinguistic communities have undertaken language ‘revitalization projects’ to fortify the social and pragmatic positions of heritage languages (for examples see: Reyhner \& Lockard 2009). However, the lack of a clear distinction between ‘language documentation’8 and ‘language development’ for so long a period of time by many practitioners leads to three observations: (1) that in general there has not been a clear distinction in the literature between language development activities and language documentation activities, by those engaged in either or both activities; (2) that in general these activities do not usually occur individually; and (3) perhaps for many language communities what is desired is not a language documentation endeavor, but rather a language development endeavor. That is, generally the activities of language development are encountered in documentation projects as persons affiliated with the academy pursue and engage minority language users. As pointed out by Mosel (2006: 68), the common case is that those activities which make it to the literature, do so because it is persons affiliated with the academy, looking to fulfill the requirements of the academy. Requirements of the academy often include publishing (Nature 2013, Priem, Taraborelli, Groth \& Neylon 2010) and a demonstrable impact (Taylor 2011) which itself is part of a larger departmental research profile (for an example see materials by: Provost of the University of Wisconsin 2014).\par{}{\needspace{3\baselineskip}
\vspace{10pt}\noindent
\fontsize{13}{15.6}\selectfont \textit{{\noindent
\raisebox{\baselineskip}[0pt]{\pdfbookmark[3]{{2.3.1 } Defining Language Development}{s1.3.1}}\raisebox{\baselineskip}[0pt]{\protect\hypertarget{s1.3.1}{}}{2.3.1 }Defining Language Development}}\markright{Defining Language Development}
\XLingPaperaddtocontents{s1.3.1}}\par{}
\penalty10000\vspace{10pt}\penalty10000{\needspace{3\baselineskip}
\vspace{15pt}\noindent
\fontsize{13}{15.6}\selectfont \textbf{{\noindent
\raisebox{\baselineskip}[0pt]{\pdfbookmark[2]{{2.4 } Writing in Society}{s1.4}}\raisebox{\baselineskip}[0pt]{\protect\hypertarget{s1.4}{}}{2.4 }Writing in Society}}\markright{Writing in Society}
\XLingPaperaddtocontents{s1.4}}\par{}
\penalty10000\vspace{10pt}\penalty10000{\needspace{3\baselineskip}
\vspace{15pt}\noindent
\fontsize{13}{15.6}\selectfont \textbf{{\noindent
\raisebox{\baselineskip}[0pt]{\pdfbookmark[2]{{2.5 } The Role and impact of design}{s.1.5}}\raisebox{\baselineskip}[0pt]{\protect\hypertarget{s.1.5}{}}{2.5 }The Role and impact of design}}\markright{The Role and impact of design}
\XLingPaperaddtocontents{s.1.5}}\par{}
\penalty10000\vspace{10pt}\penalty10000{\needspace{3\baselineskip}
\vspace{15pt}\noindent
\fontsize{13}{15.6}\selectfont \textbf{{\noindent
\raisebox{\baselineskip}[0pt]{\pdfbookmark[2]{{2.6 } The Role and impact of technical social systems}{s1.6}}\raisebox{\baselineskip}[0pt]{\protect\hypertarget{s1.6}{}}{2.6 }The Role and impact of technical social systems}}\markright{The Role and impact of technical social systems}
\XLingPaperaddtocontents{s1.6}}\par{}
\penalty10000\vspace{10pt}\penalty10000{\needspace{3\baselineskip}
\vspace{15pt}\noindent
\fontsize{13}{15.6}\selectfont \textbf{{\noindent
\raisebox{\baselineskip}[0pt]{\pdfbookmark[2]{{2.7 } The objectification of languages}{s1.7}}\raisebox{\baselineskip}[0pt]{\protect\hypertarget{s1.7}{}}{2.7 }The objectification of languages}}\markright{The objectification of languages}
\XLingPaperaddtocontents{s1.7}}\par{}
\penalty10000\vspace{10pt}\penalty10000{\needspace{3\baselineskip}
\vspace{10pt}\noindent
\fontsize{13}{15.6}\selectfont \textit{{\noindent
\raisebox{\baselineskip}[0pt]{\pdfbookmark[3]{{2.7.1 } Objectification of the language}{s1.7.1}}\raisebox{\baselineskip}[0pt]{\protect\hypertarget{s1.7.1}{}}{2.7.1 }Objectification of the language}}\markright{Objectification of the language}
\XLingPaperaddtocontents{s1.7.1}}\par{}
\penalty10000\vspace{10pt}\penalty10000{\needspace{3\baselineskip}
\vspace{10pt}\noindent
\fontsize{13}{15.6}\selectfont \textit{{\noindent
\raisebox{\baselineskip}[0pt]{\pdfbookmark[3]{{2.7.2 } Object Culture}{s1.7.2}}\raisebox{\baselineskip}[0pt]{\protect\hypertarget{s1.7.2}{}}{2.7.2 }Object Culture}}\markright{Object Culture}
\XLingPaperaddtocontents{s1.7.2}}\par{}
\penalty10000\vspace{10pt}\penalty10000\clearpage
\thispagestyle{bodyfirstpage}\markboth{Methods}{Methods}
\XLingPaperaddtocontents{c2}{\vspace*{.65in}\noindent
\fontsize{14}{16.8}\selectfont \textbf{{\centering
CHAPTER \raisebox{\baselineskip}[0pt]{\protect\hypertarget{c2}{}}\raisebox{\baselineskip}[0pt]{\pdfbookmark[1]{3 Methods}{c2}}3\\}}}\par{}
{\noindent
\fontsize{14}{16.8}\selectfont \textbf{{\centering
Methods\\}}}\par{}
\vspace{16pt}\indent This reviews the literature.\par{}{\needspace{3\baselineskip}
\vspace{15pt}\noindent
\fontsize{13}{15.6}\selectfont \textbf{{\noindent
\raisebox{\baselineskip}[0pt]{\pdfbookmark[2]{{3.1 } (Methodology) Methodological considerations}{s2.1}}\raisebox{\baselineskip}[0pt]{\protect\hypertarget{s2.1}{}}{3.1 }(Methodology) Methodological considerations}}\markright{(Methodology) Methodological considerations}
\XLingPaperaddtocontents{s2.1}}\par{}
\penalty10000\vspace{10pt}\penalty10000{\needspace{3\baselineskip}
\vspace{10pt}\noindent
\fontsize{13}{15.6}\selectfont \textit{{\noindent
\raisebox{\baselineskip}[0pt]{\pdfbookmark[3]{{3.1.1 } Keyboards}{s2.1.1}}\raisebox{\baselineskip}[0pt]{\protect\hypertarget{s2.1.1}{}}{3.1.1 }Keyboards}}\markright{Keyboards}
\XLingPaperaddtocontents{s2.1.1}}\par{}
\penalty10000\vspace{10pt}\penalty10000\indent Physical v.s virtual\par{}\indent mobile v.s stationary\par{}{\needspace{3\baselineskip}
\vspace{10pt}\noindent
\fontsize{13}{15.6}\selectfont \textit{{\noindent
\raisebox{\baselineskip}[0pt]{\pdfbookmark[3]{{3.1.2 } Orthographies}{s2.1.2}}\raisebox{\baselineskip}[0pt]{\protect\hypertarget{s2.1.2}{}}{3.1.2 }Orthographies}}\markright{Orthographies}
\XLingPaperaddtocontents{s2.1.2}}\par{}
\penalty10000\vspace{10pt}\penalty10000\indent Do they get designed around computer technology or are they\par{}{\needspace{3\baselineskip}
\vspace{10pt}\noindent
\fontsize{13}{15.6}\selectfont \textit{{\noindent
\raisebox{\baselineskip}[0pt]{\pdfbookmark[3]{{3.1.3 } Typing behaviors}{s2.1.3}}\raisebox{\baselineskip}[0pt]{\protect\hypertarget{s2.1.3}{}}{3.1.3 }Typing behaviors}}\markright{Typing behaviors}
\XLingPaperaddtocontents{s2.1.3}}\par{}
\penalty10000\vspace{10pt}\penalty10000\indent What do I mean by this? Is it typing behaviors as in use of the keyboard layout? Or what the behaviors should be like Dvorak v.s. Qwerty.\par{}{\needspace{3\baselineskip}
\vspace{10pt}\noindent
\fontsize{13}{15.6}\selectfont \textit{{\noindent
\raisebox{\baselineskip}[0pt]{\pdfbookmark[3]{{3.1.4 } Current Design Processes}{s2.1.4}}\raisebox{\baselineskip}[0pt]{\protect\hypertarget{s2.1.4}{}}{3.1.4 }Current Design Processes}}\markright{Current Design Processes}
\XLingPaperaddtocontents{s2.1.4}}\par{}
\penalty10000\vspace{10pt}\penalty10000\indent Something about different views on these relationships.\par{}\indent What is the current processes?\par{}{\needspace{3\baselineskip}
\vspace{10pt}\noindent
{\noindent
\raisebox{\baselineskip}[0pt]{\pdfbookmark[4]{{3.1.4.1 } The design of orthographies and keyboards}{s2.1.4.1}}\raisebox{\baselineskip}[0pt]{\protect\hypertarget{s2.1.4.1}{}}{3.1.4.1 }The design of orthographies and keyboards}\markright{The design of orthographies and keyboards}
\XLingPaperaddtocontents{s2.1.4.1}}\par{}
\penalty10000\vspace{10pt}\penalty10000{\needspace{3\baselineskip}
\vspace{10pt}\noindent
{\noindent
\raisebox{\baselineskip}[0pt]{\pdfbookmark[4]{{3.1.4.2 } Good Design}{s2.1.4.2}}\raisebox{\baselineskip}[0pt]{\protect\hypertarget{s2.1.4.2}{}}{3.1.4.2 }Good Design}\markright{Good Design}
\XLingPaperaddtocontents{s2.1.4.2}}\par{}
\penalty10000\vspace{10pt}\penalty10000{\needspace{3\baselineskip}
\vspace{15pt}\noindent
\fontsize{13}{15.6}\selectfont \textbf{{\noindent
\raisebox{\baselineskip}[0pt]{\pdfbookmark[2]{{3.2 } Orthography text samples and analyzed keyboard layouts}{s2.2}}\raisebox{\baselineskip}[0pt]{\protect\hypertarget{s2.2}{}}{3.2 }Orthography text samples and analyzed keyboard layouts}}\markright{Orthography text samples and analyzed keyboard layouts}
\XLingPaperaddtocontents{s2.2}}\par{}
\penalty10000\vspace{10pt}\penalty10000\clearpage
\thispagestyle{bodyfirstpage}\markboth{(Results)The data to be explored}{(Results)The data to be explored}
\XLingPaperaddtocontents{c3}{\vspace*{.65in}\noindent
\fontsize{14}{16.8}\selectfont \textbf{{\centering
CHAPTER \raisebox{\baselineskip}[0pt]{\protect\hypertarget{c3}{}}\raisebox{\baselineskip}[0pt]{\pdfbookmark[1]{4 (Results)The data to be explored}{c3}}4\\}}}\par{}
{\noindent
\fontsize{14}{16.8}\selectfont \textbf{{\centering
(Results)The data to be explored\\}}}\par{}
\vspace{16pt}\indent What is a design framework and why is it needed?\par{}\indent How does design relate to keyboards?\par{}\indent What are we teaching about language by showing complex ways to type a language?\par{}\indent The User group and the community are to separate but related concepts.\par{}{\needspace{3\baselineskip}
\vspace{15pt}\noindent
\fontsize{13}{15.6}\selectfont \textbf{{\noindent
\raisebox{\baselineskip}[0pt]{\pdfbookmark[2]{{4.1 } impacts}{s31}}\raisebox{\baselineskip}[0pt]{\protect\hypertarget{s31}{}}{4.1 }impacts}}\markright{impacts}
\XLingPaperaddtocontents{s31}}\par{}
\penalty10000\vspace{10pt}\penalty10000{\needspace{3\baselineskip}
\vspace{10pt}\noindent
\fontsize{13}{15.6}\selectfont \textit{{\noindent
\raisebox{\baselineskip}[0pt]{\pdfbookmark[3]{{4.1.1 } objectification of the orthography}{s311}}\raisebox{\baselineskip}[0pt]{\protect\hypertarget{s311}{}}{4.1.1 }objectification of the orthography}}\markright{objectification of the orthography}
\XLingPaperaddtocontents{s311}}\par{}
\penalty10000\vspace{10pt}\penalty10000\indent The mixing of the orthography and I density as a brand\par{}{\needspace{3\baselineskip}
\vspace{10pt}\noindent
\fontsize{13}{15.6}\selectfont \textit{{\noindent
\raisebox{\baselineskip}[0pt]{\pdfbookmark[3]{{4.1.2 } objectification of the keyboard, and the keyboard layout}{s312}}\raisebox{\baselineskip}[0pt]{\protect\hypertarget{s312}{}}{4.1.2 }objectification of the keyboard, and the keyboard layout}}\markright{objectification of the keyboard, and the keyboard layout}
\XLingPaperaddtocontents{s312}}\par{}
\penalty10000\vspace{10pt}\penalty10000{\needspace{3\baselineskip}
\vspace{15pt}\noindent
\fontsize{13}{15.6}\selectfont \textbf{{\noindent
\raisebox{\baselineskip}[0pt]{\pdfbookmark[2]{{4.2 } What should a keyboard layout enable people to do?}{s32}}\raisebox{\baselineskip}[0pt]{\protect\hypertarget{s32}{}}{4.2 }What should a keyboard layout enable people to do?}}\markright{What should a keyboard layout enable people to do?}
\XLingPaperaddtocontents{s32}}\par{}
\penalty10000\vspace{10pt}\penalty10000{\needspace{3\baselineskip}
\vspace{10pt}\noindent
\fontsize{13}{15.6}\selectfont \textit{{\noindent
\raisebox{\baselineskip}[0pt]{\pdfbookmark[3]{{4.2.1 } write in their language, in their script}{s321}}\raisebox{\baselineskip}[0pt]{\protect\hypertarget{s321}{}}{4.2.1 }write in their language, in their script}}\markright{write in their language, in their script}
\XLingPaperaddtocontents{s321}}\par{}
\penalty10000\vspace{10pt}\penalty10000\indent What is the difference between writing and typing?\par{}\indent Why is this significance relevant?\par{}{\needspace{3\baselineskip}
\vspace{10pt}\noindent
\fontsize{13}{15.6}\selectfont \textit{{\noindent
\raisebox{\baselineskip}[0pt]{\pdfbookmark[3]{{4.2.2 } Control the computer}{s322}}\raisebox{\baselineskip}[0pt]{\protect\hypertarget{s322}{}}{4.2.2 }Control the computer}}\markright{Control the computer}
\XLingPaperaddtocontents{s322}}\par{}
\penalty10000\vspace{10pt}\penalty10000{\needspace{3\baselineskip}
\vspace{15pt}\noindent
\fontsize{13}{15.6}\selectfont \textbf{{\noindent
\raisebox{\baselineskip}[0pt]{\pdfbookmark[2]{{4.3 } What components does the framework need to contain?}{s33}}\raisebox{\baselineskip}[0pt]{\protect\hypertarget{s33}{}}{4.3 }What components does the framework need to contain?}}\markright{What components does the framework need to contain?}
\XLingPaperaddtocontents{s33}}\par{}
\penalty10000\vspace{10pt}\penalty10000\indent Language Family\par{}\indent Language overlap settings\par{}\indent Language use in diaspora\par{}\indent Unicode and non-encode text\par{}\clearpage
\thispagestyle{bodyfirstpage}\markboth{Methodology}{Methodology}
\XLingPaperaddtocontents{c4}{\vspace*{.65in}\noindent
\fontsize{14}{16.8}\selectfont \textbf{{\centering
CHAPTER \raisebox{\baselineskip}[0pt]{\protect\hypertarget{c4}{}}\raisebox{\baselineskip}[0pt]{\pdfbookmark[1]{5 Methodology}{c4}}5\\}}}\par{}
{\noindent
\fontsize{14}{16.8}\selectfont \textbf{{\centering
Methodology\\}}}\par{}
\vspace{16pt}{\needspace{3\baselineskip}
\vspace{15pt}\noindent
\fontsize{13}{15.6}\selectfont \textbf{{\noindent
\raisebox{\baselineskip}[0pt]{\pdfbookmark[2]{{5.1 } UX Analysis}{s41}}\raisebox{\baselineskip}[0pt]{\protect\hypertarget{s41}{}}{5.1 }UX Analysis}}\markright{UX Analysis}
\XLingPaperaddtocontents{s41}}\par{}
\penalty10000\vspace{10pt}\penalty10000\indent Provide a definition of UX\par{}\indent Provide relevance to of UX decisions to Linguistics and language choice\par{}{\needspace{3\baselineskip}
\vspace{15pt}\noindent
\fontsize{13}{15.6}\selectfont \textbf{{\noindent
\raisebox{\baselineskip}[0pt]{\pdfbookmark[2]{{5.2 } Methods in UX analysis}{s42}}\raisebox{\baselineskip}[0pt]{\protect\hypertarget{s42}{}}{5.2 }Methods in UX analysis}}\markright{Methods in UX analysis}
\XLingPaperaddtocontents{s42}}\par{}
\penalty10000\vspace{10pt}\penalty10000\indent Some general methods in UX analysis\par{}\indent Do linguistics do UX analysis?\par{}{\needspace{3\baselineskip}
\vspace{10pt}\noindent
\fontsize{13}{15.6}\selectfont \textit{{\noindent
\raisebox{\baselineskip}[0pt]{\pdfbookmark[3]{{5.2.1 } Specific methods related the acquisition of my data}{s421}}\raisebox{\baselineskip}[0pt]{\protect\hypertarget{s421}{}}{5.2.1 }Specific methods related the acquisition of my data}}\markright{Specific methods related the acquisition of my data}
\XLingPaperaddtocontents{s421}}\par{}
\penalty10000\vspace{10pt}\penalty10000{\needspace{3\baselineskip}
\vspace{10pt}\noindent
{\noindent
\raisebox{\baselineskip}[0pt]{\pdfbookmark[4]{{5.2.1.1 } Keystroke Counting}{s4211}}\raisebox{\baselineskip}[0pt]{\protect\hypertarget{s4211}{}}{5.2.1.1 }Keystroke Counting}\markright{Keystroke Counting}
\XLingPaperaddtocontents{s4211}}\par{}
\penalty10000\vspace{10pt}\penalty10000\indent Character counting v.s keystroke counting\par{}{\needspace{3\baselineskip}
\vspace{10pt}\noindent
{\noindent
\raisebox{\baselineskip}[0pt]{\pdfbookmark[4]{{5.2.1.2 } Survey Data}{s4212}}\raisebox{\baselineskip}[0pt]{\protect\hypertarget{s4212}{}}{5.2.1.2 }Survey Data}\markright{Survey Data}
\XLingPaperaddtocontents{s4212}}\par{}
\penalty10000\vspace{10pt}\penalty10000\indent The questions asked in the survey\par{}\indent The reason why the questions are asked\par{}{\needspace{3\baselineskip}
\vspace{15pt}\noindent
\fontsize{13}{15.6}\selectfont \textbf{{\noindent
\raisebox{\baselineskip}[0pt]{\pdfbookmark[2]{{5.3 } The Role of linguistic knowledge in UX}{s44}}\raisebox{\baselineskip}[0pt]{\protect\hypertarget{s44}{}}{5.3 }The Role of linguistic knowledge in UX}}\markright{The Role of linguistic knowledge in UX}
\XLingPaperaddtocontents{s44}}\par{}
\penalty10000\vspace{10pt}\penalty10000\indent {ux}\par{}\indent {adsf{\hyperlink{rBirkenSilverman1997TheRo}{{}}}uxa}\par{}\clearpage
\thispagestyle{bodyfirstpage}\markboth{The results of several languages}{The results of several languages}
\XLingPaperaddtocontents{c5}{\vspace*{.65in}\noindent
\fontsize{14}{16.8}\selectfont \textbf{{\centering
CHAPTER \raisebox{\baselineskip}[0pt]{\protect\hypertarget{c5}{}}\raisebox{\baselineskip}[0pt]{\pdfbookmark[1]{6 The results of several languages}{c5}}6\\}}}\par{}
{\noindent
\fontsize{14}{16.8}\selectfont \textbf{{\centering
The results of several languages\\}}}\par{}
\vspace{16pt}{\needspace{3\baselineskip}
\vspace{15pt}\noindent
\fontsize{13}{15.6}\selectfont \textbf{{\noindent
\raisebox{\baselineskip}[0pt]{\pdfbookmark[2]{{6.1 } Use Case \#1 Me'phaa}{s51}}\raisebox{\baselineskip}[0pt]{\protect\hypertarget{s51}{}}{6.1 }Use Case \#1 Me'phaa}}\markright{Use Case \#1 Me'phaa}
\XLingPaperaddtocontents{s51}}\par{}
\penalty10000\vspace{10pt}\penalty10000{\needspace{3\baselineskip}
\vspace{10pt}\noindent
\fontsize{13}{15.6}\selectfont \textit{{\noindent
\raisebox{\baselineskip}[0pt]{\pdfbookmark[3]{{6.1.1 } Phonology}{s511}}\raisebox{\baselineskip}[0pt]{\protect\hypertarget{s511}{}}{6.1.1 }Phonology}}\markright{Phonology}
\XLingPaperaddtocontents{s511}}\par{}
\penalty10000\vspace{10pt}\penalty10000{\needspace{3\baselineskip}
\vspace{10pt}\noindent
\fontsize{13}{15.6}\selectfont \textit{{\noindent
\raisebox{\baselineskip}[0pt]{\pdfbookmark[3]{{6.1.2 } Orthography}{s512}}\raisebox{\baselineskip}[0pt]{\protect\hypertarget{s512}{}}{6.1.2 }Orthography}}\markright{Orthography}
\XLingPaperaddtocontents{s512}}\par{}
\penalty10000\vspace{10pt}\penalty10000{\needspace{3\baselineskip}
\vspace{10pt}\noindent
\fontsize{13}{15.6}\selectfont \textit{{\noindent
\raisebox{\baselineskip}[0pt]{\pdfbookmark[3]{{6.1.3 } Keyboard Layout}{s513}}\raisebox{\baselineskip}[0pt]{\protect\hypertarget{s513}{}}{6.1.3 }Keyboard Layout}}\markright{Keyboard Layout}
\XLingPaperaddtocontents{s513}}\par{}
\penalty10000\vspace{10pt}\penalty10000{\needspace{3\baselineskip}
\vspace{10pt}\noindent
\fontsize{13}{15.6}\selectfont \textit{{\noindent
\raisebox{\baselineskip}[0pt]{\pdfbookmark[3]{{6.1.4 } Social Use setting of typing in the language}{s514}}\raisebox{\baselineskip}[0pt]{\protect\hypertarget{s514}{}}{6.1.4 }Social Use setting of typing in the language}}\markright{Social Use setting of typing in the language}
\XLingPaperaddtocontents{s514}}\par{}
\penalty10000\vspace{10pt}\penalty10000{\needspace{3\baselineskip}
\vspace{10pt}\noindent
\fontsize{13}{15.6}\selectfont \textit{{\noindent
\raisebox{\baselineskip}[0pt]{\pdfbookmark[3]{{6.1.5 } Statistics from Text Analysis}{s515}}\raisebox{\baselineskip}[0pt]{\protect\hypertarget{s515}{}}{6.1.5 }Statistics from Text Analysis}}\markright{Statistics from Text Analysis}
\XLingPaperaddtocontents{s515}}\par{}
\penalty10000\vspace{10pt}\penalty10000{\needspace{3\baselineskip}
\vspace{10pt}\noindent
\fontsize{13}{15.6}\selectfont \textit{{\noindent
\raisebox{\baselineskip}[0pt]{\pdfbookmark[3]{{6.1.6 } Observations}{s516}}\raisebox{\baselineskip}[0pt]{\protect\hypertarget{s516}{}}{6.1.6 }Observations}}\markright{Observations}
\XLingPaperaddtocontents{s516}}\par{}
\penalty10000\vspace{10pt}\penalty10000{\needspace{3\baselineskip}
\vspace{15pt}\noindent
\fontsize{13}{15.6}\selectfont \textbf{{\noindent
\raisebox{\baselineskip}[0pt]{\pdfbookmark[2]{{6.2 } Use Case \#2 Chinantec}{s52}}\raisebox{\baselineskip}[0pt]{\protect\hypertarget{s52}{}}{6.2 }Use Case \#2 Chinantec}}\markright{Use Case \#2 Chinantec}
\XLingPaperaddtocontents{s52}}\par{}
\penalty10000\vspace{10pt}\penalty10000{\needspace{3\baselineskip}
\vspace{15pt}\noindent
\fontsize{13}{15.6}\selectfont \textbf{{\noindent
\raisebox{\baselineskip}[0pt]{\pdfbookmark[2]{{6.3 } Use Case \#3 Spanish}{s53}}\raisebox{\baselineskip}[0pt]{\protect\hypertarget{s53}{}}{6.3 }Use Case \#3 Spanish}}\markright{Use Case \#3 Spanish}
\XLingPaperaddtocontents{s53}}\par{}
\penalty10000\vspace{10pt}\penalty10000{\needspace{3\baselineskip}
\vspace{15pt}\noindent
\fontsize{13}{15.6}\selectfont \textbf{{\noindent
\raisebox{\baselineskip}[0pt]{\pdfbookmark[2]{{6.4 } Use Case \#4 English}{s54}}\raisebox{\baselineskip}[0pt]{\protect\hypertarget{s54}{}}{6.4 }Use Case \#4 English}}\markright{Use Case \#4 English}
\XLingPaperaddtocontents{s54}}\par{}
\penalty10000\vspace{10pt}\penalty10000{\needspace{3\baselineskip}
\vspace{15pt}\noindent
\fontsize{13}{15.6}\selectfont \textbf{{\noindent
\raisebox{\baselineskip}[0pt]{\pdfbookmark[2]{{6.5 } Use Case \#5 ??? - from Africa}{s55}}\raisebox{\baselineskip}[0pt]{\protect\hypertarget{s55}{}}{6.5 }Use Case \#5 ??? - from Africa}}\markright{Use Case \#5 ??? - from Africa}
\XLingPaperaddtocontents{s55}}\par{}
\penalty10000\vspace{10pt}\penalty10000\clearpage
\thispagestyle{bodyfirstpage}\markboth{What we can observe from these Use Cases and layouts}{What we can observe from these Use Cases and layouts}
\XLingPaperaddtocontents{c6}{\vspace*{.65in}\noindent
\fontsize{14}{16.8}\selectfont \textbf{{\centering
CHAPTER \raisebox{\baselineskip}[0pt]{\protect\hypertarget{c6}{}}\raisebox{\baselineskip}[0pt]{\pdfbookmark[1]{7 What we can observe from these Use Cases and layouts}{c6}}7\\}}}\par{}
{\noindent
\fontsize{14}{16.8}\selectfont \textbf{{\centering
What we can observe from these Use Cases and layouts\\}}}\par{}
\vspace{16pt}\pagestyle{body}\clearpage
\thispagestyle{empty}{\needspace{3\baselineskip}
\clearpage
\vspace*{4.075in}\noindent
{\centering
\raisebox{\baselineskip}[0pt]{\protect\hypertarget{rXLingPapAppendiciesPage}{}}APPENDICES\\}\markboth{APPENDICES}{APPENDICES}
\XLingPaperaddtocontents{rXLingPapAppendiciesPage}}\penalty10000\par{}
\vfil
\raisebox{\baselineskip}[0pt]{\pdfbookmark[1]{APPENDICES}{rXLingPapAppendiciesPage}}\clearpage
\thispagestyle{bodyfirstpage}\markboth{Appendix I: Glossary of technical concepts and terms}{Appendix I: Glossary of technical concepts and terms}
\XLingPaperaddtocontents{a}{\vspace*{.65in}\noindent
{\centering
APPENDIX \raisebox{\baselineskip}[0pt]{\protect\hypertarget{a}{}}\raisebox{\baselineskip}[0pt]{\pdfbookmark[1]{A Appendix I: Glossary of technical concepts and terms}{a}}A\\}}\par{}
{\noindent
{\centering
Appendix I: Glossary of technical concepts and terms\\}}\par{}
\vspace{16pt}\clearpage
\thispagestyle{bodyfirstpage}\markboth{Appendix II: List and purpose of referenced standards}{Appendix II: List and purpose of referenced standards}
\XLingPaperaddtocontents{b}{\vspace*{.65in}\noindent
{\centering
APPENDIX \raisebox{\baselineskip}[0pt]{\protect\hypertarget{b}{}}\raisebox{\baselineskip}[0pt]{\pdfbookmark[1]{B Appendix II: List and purpose of referenced standards}{b}}B\\}}\par{}
{\noindent
{\centering
Appendix II: List and purpose of referenced standards\\}}\par{}
\vspace{16pt}\clearpage
\thispagestyle{bodyfirstpage}\markboth{Appendix III: Full text of analyzed text}{Appendix III: Full text of analyzed text}
\XLingPaperaddtocontents{aAnalyzedTexts}{\vspace*{.65in}\noindent
{\centering
APPENDIX \raisebox{\baselineskip}[0pt]{\protect\hypertarget{aAnalyzedTexts}{}}\raisebox{\baselineskip}[0pt]{\pdfbookmark[1]{C Appendix III: Full text of analyzed text}{aAnalyzedTexts}}C\\}}\par{}
{\noindent
{\centering
Appendix III: Full text of analyzed text\\}}\par{}
\vspace{16pt}{\needspace{3\baselineskip}
\vspace{15pt}\noindent
\fontsize{13}{15.6}\selectfont \textbf{{\noindent
\raisebox{\baselineskip}[0pt]{\pdfbookmark[2]{{C.1 } Meꞌphaa Full Text}{s-tcf}}\raisebox{\baselineskip}[0pt]{\protect\hypertarget{s-tcf}{}}{C.1 }Meꞌphaa Full Text}}\markright{Meꞌphaa Full Text}
\XLingPaperaddtocontents{s-tcf}}\par{}
\penalty10000\vspace{10pt}\penalty10000\needspace{5\baselineskip}

\penalty-3000
\begin{description}
\setlength{\topsep}{0pt}\setlength{\partopsep}{0pt}\setlength{\itemsep}{0pt}\setlength{\parsep}{0pt}\setlength{\parskip}{0pt}\setlength{\leftmargini}{1em}\setlength{\leftmarginii}{1em}\setlength{\leftmarginiii}{1em}\setlength{\leftmarginiv}{1em}\penalty10000\item[ISO 639-3 code of language:]{\textsquarebracketleft{}tcf\textsquarebracketright{}}
\penalty10000\item[Title of the text:]{Santiago (Me̱ꞌpha̱a̱ Mañúwi̱ín)}
\needspace{5\baselineskip}

\penalty-3000\item[Cited as:]{Carrasco Zúñiga, Estanislao \& Mark L. Weathers. 2008-2010. Santiago (James). Ms., Pre-Publication Draft of Bible Portion.}
\needspace{5\baselineskip}

\penalty-3000\item[Text provenance:]{The text was received from the Me̱ꞌpha̱a language development and Bible translation team via Mark L. Weathers on 31 May 2011.}
\needspace{5\baselineskip}

\penalty-3000\item[What I did to the text before using it in comparisons:]{this text was likely the most complex to process. It required conversion from a custom encoding to Unicode. \_\_(tool used; mapping uses; method obtained)\_\_*since the team has left the moved their project to unicode* after conversion, SFM markers were removed. Section headers were removed. Carriage returns were also removed.}
\penalty10000\item[Copyright holder as indicated:]{SIL International and local language speakers. Used by written permission.}
\penalty10000\item[The Text:]{Ikhúún ñajunꞌ Santiágo̱, mbo̱ naꞌne ñajuun Ana̱ꞌlóꞌ jamí Táta̱ Jesukrísto̱. Na̱xu̱ꞌmá i̱yi̱i̱ꞌ ríge̱ꞌ inalaꞌ ikháanꞌ tsáanꞌ mbo̱ gu̱wa̱ꞌ a̱jma̱ múú kuthiin i̱ji̱in Israél ñajwanlaꞌ, tsí ni̱drúꞌúún mbá xúgíí inuu numbaa. Na̱ra̱xa̱ánꞌlaꞌ. A̱ngui̱nꞌ, tsáanꞌ ninimba̱ꞌlaꞌ ju̱ya̱á Jesús, ga̱ju̱ma̱ꞌlaꞌ rí phú gagi juwalaꞌ ído̱ rí nanújngalaꞌ awúun mbaꞌa inii gajmá. Numuu ndu̱ya̱á málaꞌ rí ído̱ rí na̱ꞌnga̱ꞌlaꞌ inuu gajmá, nasngájma ne̱ rí gakon rí jañii a̱kia̱nꞌlaꞌ ju̱ya̱á Ana̱ꞌlóꞌ, jamí naꞌne ne̱ rí ma̱wajún gúkuálaꞌ. I̱ndo̱ó máꞌ gíꞌmaa rí ma̱wajún gúkuálaꞌ xúgíí mbiꞌi, kajngó ma̱jráanꞌlaꞌ jamí ma̱ꞌne rí jañii a̱kia̱nꞌlaꞌ, asndo rí náxáꞌyóo nitháan rí jaꞌyoo ma̱nindxa̱ꞌlaꞌ. Xí mbáa tsí ikháanꞌlaꞌ tsíꞌyoo̱ dí gáꞌnii̱, gaꞌthán jáñuu̱ Ana̱ꞌlóꞌ, jamí Ana̱ꞌlóꞌ gáꞌne rí mba̱ꞌyoo̱ rí ma̱ꞌnii̱. Numuu rí ikhaa̱ tsígéwee̱n rí naxnúu̱ mbá xúgíin tsí nu̱nda̱ꞌáa̱ jamí tsíꞌthée̱n numuu ne̱. I̱ndo̱ó máꞌ numuu rí tsí na̱nda̱ꞌa̱, ga̱nda̱ꞌe̱e̱ ga̱jma̱á mbá jañii a̱kuii̱n, ma̱xáꞌne rí a̱jma̱ a̱kuii̱n nitháan. Numuu rí tsí a̱jma̱ a̱kui̱in asndo xó rí nambúxu̱u̱ꞌ inuu iya a̱pha̱ jaꞌnii̱, rí ge̱e̱ ñúꞌú ixpátraꞌa eꞌne gíñá. Xa̱bo̱ tsí xkuaꞌnii jaꞌnii, xáju̱mu̱u̱ rí ma̱janáa̱ tháan rí na̱nda̱ꞌe̱e̱ gáꞌne Ana̱ꞌlóꞌ, numuu rí tsí a̱jma̱ a̱kui̱in, nariꞌkhuu̱ máꞌ xúꞌkhue̱n mbá xúgíí rí naꞌnii̱. Dxájwalóꞌ tsí ngínáa, xáti̱yu̱u̱ꞌ, numuu rí phú gíꞌdoo numuu̱ ná inuu Ana̱ꞌlóꞌ, xómáꞌ tsí phú gíꞌdoo rá, xáti̱yu̱u̱ꞌ ído̱ rí Ana̱ꞌlóꞌ maxríguíi̱, numuu rí tsí phú gíꞌdoo na̱nguá mba̱yo̱ꞌ xtáa̱ xómá re̱ꞌe̱ rí ríga̱ ná xanáá. Ído̱ rí na̱ꞌkha̱ a̱kha̱ꞌ, nagigoo numbaa, jamí nojndoo iná. A̱ꞌkhue̱n rí nafrigu re̱ꞌe̱ jamí nánguá mitsaan giaxu̱u̱ ne̱. Xkuaꞌnii máꞌ ma̱mbáa tsí phú gíꞌdoo jamí ma̱mbá ri̱ga̱a mbá xúgíí rí gíꞌdoo̱ mangaa. Phú gagi xtáa xa̱bo̱ tsí tsíꞌne rí xkawe̱ꞌ ído̱ rí na̱gu̱ma rájáa, numuu ído̱ rí ni̱ꞌngo̱o̱ inuu gajmá, a̱ꞌkhue̱n mbayáa̱ numuu̱ rí ma̱xtáa̱ jámuu, rí nixuda mina̱ꞌ Ana̱ꞌlóꞌ rí maxnúu̱ tsí nandúún juyáa̱. Ído̱ rí mbáa tsí ikháanꞌlaꞌ na̱gu̱ma rájáa rí ma̱ꞌne rí xkawe̱ꞌ, ma̱xáꞌthée̱n rí Ana̱ꞌlóꞌ neꞌne rájáa̱. Numuu rí tsí Ana̱ꞌlóꞌ tsíyoo rí ma̱gu̱ma rájáa̱ rí ma̱ꞌnii̱ rí xkawe̱ꞌ, ni máꞌ ikhaa̱ tsíꞌne rájáa̱ nimbáa. A̱ꞌkhue̱n gakon, mbámbáa na̱gu̱ma rájáa ído̱ rí nixmángua̱ꞌa̱a̱n eꞌne rí xkawe̱ꞌ nandxaꞌwá mine̱e̱ꞌ jamí rí na̱ni̱gu̱u̱ꞌ. Ído̱ rí ni̱to̱ꞌo̱o̱ máꞌ ju̱ma̱ rí xkawe̱ꞌ ná idxu̱u̱ xa̱bo̱ rá, ma̱ga̱nú mbiꞌi rí ma̱ꞌnii̱ aꞌkhán gáꞌne ne̱. Jamí ído̱ rí wámba̱ máꞌ nigaja̱a̱ aꞌkhán rá, ma̱ja̱ñúu̱ gáꞌne ne̱. A̱ngui̱nꞌ, phú nandoꞌ ja̱ya̱ꞌlaꞌ. Xángra̱ꞌáanꞌlaꞌ. Xúgíí kixná rí phú máján jamí xtamínu̱ꞌ rí jañii wáa na̱ꞌkha̱ ná mikhuíí, naxná ne̱ Ana̱ꞌlóꞌ tsí neꞌne ku̱mii a̱ꞌgua̱án. Tsí Ana̱ꞌlóꞌ nimiꞌtsú tsíxtiꞌkhuu xó jaꞌnii̱ xómá a̱ꞌgua̱án, ni máꞌ tsíba̱ñi̱i̱ꞌ xómá nákua. Ana̱ꞌlóꞌ neꞌne rí ma̱gu̱máaꞌlóꞌ mbu̱jú ga̱jma̱á majñu̱u̱ ajngáa gakon, numuu rí xkuaꞌnii ndiyoo̱ ikhaa̱, kajngó ma̱nindxa̱ꞌlóꞌ i̱jii̱n tsí ginuu jayu. Ikhaa numuu rúꞌkhue̱n, a̱ngui̱nꞌ, tsáanꞌ phú nandoꞌ ja̱ya̱ꞌlaꞌ, xúgiáanꞌ ikháanꞌlaꞌ gíꞌmaa rí mu̱ꞌgíi ñaꞌwanlaꞌ kajngó mu̱dxawíínlaꞌ rí nithánlaꞌ. Xáꞌcha̱ꞌlaꞌ guéño rí mu̱tha̱nlaꞌ, jamí xájiꞌnáaꞌlaꞌ nacha̱ guéño. Numuu rí xa̱bo̱ tsí najiꞌnáa guéño, tsínii̱ rí máján xómá rí nandoo Ana̱ꞌlóꞌ. Ikhaa jngó, gu̱ni̱ꞌñáá ro̱ne̱laꞌ mbá xúgíí inii rí xkawe̱ꞌ, rí gatíí guéño ná a̱kia̱nꞌlaꞌ. Jamí go̱ne̱ waba mijnálaꞌ kajngó ma̱goo mu̱drígúlaꞌ ajngáa rawuun Ana̱ꞌlóꞌ rí waꞌdu máꞌ ná xo̱xta̱ꞌlaꞌ, rí gíꞌdoo tsiakhe̱ rí ma̱ꞌne jríña̱áꞌlaꞌ. Ra̱ꞌkhá mbájndi máꞌ i̱ndo̱ó rí mu̱dxawíínlaꞌ ajngóo Ana̱ꞌlóꞌ, rí gíꞌmaa rí mo̱ne̱ mbáníílaꞌ ne̱ mangaa. Xí i̱ndo̱ó máꞌ nu̱dxawíínlaꞌ jamí tsíne̱ mbáníílaꞌ, nu̱ne̱ nduwa mijná máꞌ ikháanꞌlaꞌ. Numuu rí tsí i̱ndo̱ó máꞌ nadxawuun ajngóo Ana̱ꞌlóꞌ jamí tsíꞌne mbánuu̱ ne̱, ikhaa̱ jaꞌnii̱ xómá xa̱bo̱ tsí i̱ndo̱ó máꞌ nayaxu̱u̱ inuu ná iya niwan, jamí ído̱ wámbo̱o̱ niyaxe̱ míne̱e̱ꞌ, a̱ꞌkhue̱n na̱ke̱e̱ jamí nacha̱ máꞌ imbumuu̱ xáne jaꞌnii inuu̱. Xómáꞌ tsí naꞌgíi idxu̱u̱ rí madxawuun mújúun ajngóo Ana̱ꞌlóꞌ rí phú máján, rí naꞌne jáwáanꞌlóꞌ rá, phú gagi gáxtáa̱ asndo náá máꞌ rí gáꞌnii̱, xí tsímbumuu̱ rí naꞌthán ne̱, jamí naꞌne mbánuu̱ máꞌ xúꞌkhue̱n rí nidxawuu̱n. Xí mbáa na̱ju̱mu̱u̱ rí phú máján xó rí naꞌne mba̱a̱ Ana̱ꞌlóꞌ jamí tsíña̱wu̱u̱n rawuu̱n, naꞌne nduwa mine̱e̱ꞌ máꞌ ikhaa̱, jamí nda̱a̱ nitháan numuu rí naꞌne mba̱a̱ Ana̱ꞌlóꞌ. Tsí gakon rí máján xó rí naꞌne mba̱a̱ Ana̱ꞌlóꞌ, jamí naꞌne rí na̱ni̱gu̱u̱ꞌ Tátiálóꞌ Mikhuíí ñajuun tsíge̱ꞌ: tsí nambáñúú i̱jínxuáꞌa jamí go̱ꞌóxuáꞌa ído̱ rí ndaꞌñúu̱. Xúꞌkhue̱n máꞌ rí na̱ña̱wa̱n míne̱e̱ꞌ rí ma̱xáꞌnii̱ rí xkawe̱ꞌ rí none̱ xa̱bo̱ ná numbaa ríge̱ꞌ. A̱ngui̱nꞌ, tsáanꞌ nanimba̱ꞌlaꞌ ju̱ya̱á Tátiálóꞌ Jesukrísto̱, tsí phú i̱tha̱án gíꞌdoo numuu, ra̱gíꞌmaa rí mu̱raꞌwíinlaꞌ xa̱bo̱ tsí mo̱ne̱ nga̱ju̱únlaꞌ. Ga̱ju̱ma̱ꞌlaꞌ rí ná na̱gi̱mbáanꞌlaꞌ a̱ꞌkhue̱n i̱ga̱nú mbáa xa̱bo̱ tsí phú gíꞌdoo, gída̱ꞌ ajwa̱nꞌ mo̱jmo̱ꞌ jndi ná ñawúu̱n jamí phú mitsaan xtíñuu̱. Awúun máꞌ rúꞌkhue̱n i̱ga̱nú mbáa xa̱bo̱ tsí ngínáa mangaa, júwuu̱n xtíin rí phú wayuu. Jamí ga̱ju̱ma̱ꞌlaꞌ rí phú no̱ne̱ nga̱jwa̱álaꞌ tsí mitsaan xtíñuu jamí nu̱tháa̱nlaꞌ: \textless{}\textless{}Táta̱, araꞌún ná xíle̱ rí máján wáa ge̱jyoꞌ\textgreater{}\textgreater{}, xómáꞌ tsí ngínáa nu̱tha̱ánlaꞌ: \textless{}\textless{}Ikháán, ariajún máꞌ a̱ꞌkhue̱n o araꞌún mbayíí mbo̱ꞌ\textgreater{}\textgreater{}. Á ra̱ꞌkhá nu̱raꞌwíinlaꞌ xa̱bo̱ tsí mo̱ne̱ nga̱ju̱únlaꞌ rí xkuaꞌnii e̱ne̱laꞌ rá. Jamí nanindxa̱ꞌlaꞌ xómá xa̱bo̱ ñajun tsí tsíra̱jwa̱ꞌ mbéꞌtháán ga̱jma̱á numúú mbá xúgíin xa̱bo̱. A̱ngui̱nꞌ, tsáanꞌ phú nandoꞌ ja̱ya̱ꞌlaꞌ, gu̱dxawíínlaꞌ rí ma̱tha̱nꞌlaꞌ: Ana̱ꞌlóꞌ niraꞌwíin xa̱bo̱ tsí ngíníi ná numbaa ríge̱ꞌ, kajngó ma̱nindxúu̱n xa̱bo̱ tsí phú nanimbu̱ún juyáá Jesukrísto̱, jamí rí ma̱nújngáa̱n ma̱nindxúu̱n xa̱bo̱ tsí naꞌthán ñajúún Ana̱ꞌlóꞌ, rí nixuda mine̱e̱ꞌ máꞌ ga̱jma̱á numúú tsí nandúún juyáa̱. Xómáꞌ ikháanꞌlaꞌ, nu̱ñu̱úlaꞌ tsí ngíníi asndo xó rí nda̱a̱ numúu̱. Á ra̱ꞌkháa xa̱bo̱ tsí phú guáꞌdáá none̱ ngínáaꞌlaꞌ jamí nagó judáanꞌlaꞌ ga̱jma̱á tsiakhe̱ ná guꞌwá ñajun rá dxe̱ꞌ. Á ra̱ꞌkhá ikhii̱n nuthan xkawi̱íꞌ mbiꞌyuu Jesús tsí ñajuun ña̱ꞌñalaꞌ rá dxe̱ꞌ. Phú máján máꞌ e̱ne̱laꞌ xí gakon rí no̱ne̱ mbáníílaꞌ xtángoo rí phú i̱tha̱án gíꞌdoo numuu, xómá ka̱ma naꞌthán ná ajngáa rawuun Ana̱ꞌlóꞌ: \textless{}\textless{}A̱gaaꞌ xtayáá xa̱bo̱ numbaa ga̱jma̱a̱ꞌ xómá nandaaꞌ xtaya mina̱ꞌ ikháán.\textgreater{}\textgreater{} Jamí xí nu̱raꞌwíinlaꞌ xa̱bo̱ tsí mo̱ne̱ nga̱ju̱únlaꞌ rá, gíꞌmáaꞌlaꞌ aꞌkhán ná inuu Ana̱ꞌlóꞌ, numuu rí tsíne̱ mbáníílaꞌ xtángoo rúꞌkhue̱n. Numuu rí tsí na̱ju̱mu̱u̱ rí naꞌne mbánuu xúgíí xtángawoo Ana̱ꞌlóꞌ jamí tsíꞌne mbánuu̱ maske asndo mbóó rí naꞌthán ne̱, gíꞌmaa̱ aꞌkhán rí tsíꞌne mbánuu̱ ne̱ xúgíí. Numuu rí Ana̱ꞌlóꞌ niꞌthán: \textless{}\textless{}Xátha̱ba̱a̱ꞌ ga̱jma̱a̱ꞌ tsí ra̱ꞌkháa a̱ꞌgia̱aꞌ o a̱jmba̱aꞌ ñajuun\textgreater{}\textgreater{}, jamí ikhaa̱ máꞌ niꞌthée̱n mangaa: \textless{}\textless{}Xáta̱xíya̱a̱ xa̱bo̱.\textgreater{}\textgreater{} Kajngó xí tsíthabáaꞌ ga̱jma̱a̱ꞌ tsí ra̱ꞌkháa a̱ꞌgia̱aꞌ o a̱jmba̱aꞌ ñajuun, jamí nataxíya̱a̱ xa̱bo̱ rá, gíꞌmáá aꞌkhán rí tsíthane̱ mbáníí xtángawoo Ana̱ꞌlóꞌ. Ikhaa jngó, gu̱tha̱nlaꞌ jamí ga̱juwalaꞌ xómá gíꞌmaa rí ma̱ju̱wá tsí mi̱tra̱jwa̱ꞌ numúú ga̱jma̱á mbá xtángoo rí naꞌne jáwíin xa̱bo̱ ná awúun aꞌkhán. Numuu rí ído̱ gára̱jwa̱ꞌ Ana̱ꞌlóꞌ ga̱jma̱á numúú xa̱bo̱, ma̱ñáwíin a̱kuii̱n jaꞌyoo̱ xa̱bo̱ tsí niñáwíin a̱kui̱in jaꞌñúú xa̱bo̱. Tsí nañáwíin a̱kui̱in, ma̱ꞌngo̱o̱ ído̱ rí mi̱tra̱jwa̱ꞌ numuu̱. A̱ngui̱nꞌ, tsáanꞌ nanimba̱ꞌlaꞌ ju̱ya̱á Jesús, xí mbáa naꞌthán: \textless{}\textless{}Ikhúún nanimbo̱ꞌ jaꞌyoo Ana̱ꞌlóꞌ\textgreater{}\textgreater{}, jamí nda̱a̱ nitháan rí máján iꞌnii̱, náá lá gámbáyúu̱ rúꞌkhue̱n rá. Á ma̱ꞌngo̱o̱ máꞌ rí xkuaꞌnii inimbo̱o̱ꞌ ma̱ꞌne jríya̱a̱ꞌ rá dxe̱ꞌ. Ga̱ju̱ma̱ꞌlaꞌ rí mbáa dxájwalóꞌ, xa̱biya̱ o a̱ꞌgo̱, nda̱a̱ xtíñuu̱ jamí nda̱a̱ rí mi̱khui̱i̱ tsitsíí, jamí mbáa tsí ikháanꞌlaꞌ naꞌthúu̱n: \textless{}\textless{}Ana̱ꞌlóꞌ gáꞌne tsakun rámáá ná mi̱dxu̱uꞌ. Athúwaanꞌ xtíñaaꞌ rí mika wáa jamí atseꞌtsolá ma̱gi̱ꞌmaaꞌ.\textgreater{}\textgreater{} Nda̱a̱ rí gámbáyúu̱ ajngáa rúꞌkhue̱n, xí tsíxnúu̱ rí ndaꞌyóo̱. Xkuaꞌnii máꞌ mangaa, tsí i̱ndo̱ó máꞌ naꞌthán rí nanimbo̱o̱ꞌ jaꞌyoo Ana̱ꞌlóꞌ jamí tsíꞌnii̱ rí máján, nda̱a̱ mbá jayu máꞌ numuu naꞌthée̱n rí nanimbo̱o̱ꞌ xúꞌkhue̱n rá. Ágáꞌne xí mbáa maꞌthán: \textless{}\textless{}Tikhuun nanimbu̱ún juyáá Ana̱ꞌlóꞌ, xómáꞌ i̱ꞌwíin none̱ rí máján.\textgreater{}\textgreater{} Xómáꞌ ikhúún na̱tha̱nlo̱ꞌ: Ikháán ma̱xáxóo ma̱ta̱snga̱jmúꞌ rí ninimba̱a̱ꞌ xtayáá Ana̱ꞌlóꞌ xí tsíthane̱ rí máján. Xómáꞌ ikhúún ma̱goo ma̱snga̱jmáaꞌ rí nanimbo̱ꞌ ja̱yo̱o̱ Ana̱ꞌlóꞌ ga̱jma̱á majñu̱u̱ rí máján na̱ne̱lo̱ꞌ. Khá nanimba̱a̱ꞌ máꞌ rí mbáwíi tsí ñajuun Ana̱ꞌlóꞌ xtáa rá. Phú máján máꞌ ithane̱ rí xkuaꞌnii rá, asndo gíñá guéen máꞌ nanimbu̱ún rí xkuaꞌnii mangiin, jamí asndo naguaꞌii̱n rí namíñúu̱. Xánindxa̱a̱ꞌ xa̱bo̱ tsí júgoo inuu. Á nandaaꞌ ma̱snga̱jmáaꞌ rí gakon rí ra̱gíꞌdoo numuu rí nanimbo̱o̱ꞌ xa̱bo̱ tsí nda̱a̱ rí máján eꞌne dxe̱ꞌ. Nákhí rí táta̱ xi̱ꞌñálóꞌ Abra̱ám nixnáxii̱ a̱dée̱ Isáák ná tsu̱du̱u̱ ja̱rngo̱xe̱ itsí, a̱ꞌkhue̱n niꞌthán Ana̱ꞌlóꞌ rí xa̱bo̱ tsí májáan a̱kui̱in ñajuu̱n. Nakujma nguáná máꞌ mbu̱ꞌyáálóꞌ rí Abra̱ám nisngájmee̱ rí nanimbo̱o̱ꞌ jaꞌyoo̱ Ana̱ꞌlóꞌ ga̱jma̱á majñu̱u̱ rí niꞌnii̱. Jamí ga̱jma̱á majñu̱u̱ rí niꞌnii̱, nigaja̱a̱ asndo nijráꞌáan rí jañii a̱kuii̱n ná inuu Ana̱ꞌlóꞌ. Xkuaꞌnii nimbánuu ajngáa rawuun Ana̱ꞌlóꞌ rí naꞌthán: \textless{}\textless{}Ninimbo̱o̱ꞌ Abra̱ám jaꞌyoo Ana̱ꞌlóꞌ, rúꞌkhue̱n jngó niꞌthán Ana̱ꞌlóꞌ rí májáan a̱kuii̱n.\textgreater{}\textgreater{} Jamí neꞌne mbiꞌyuu̱ \textless{}\textless{}Iyanga̱jwe̱e Ana̱ꞌlóꞌ\textgreater{}\textgreater{}. Kajngó ga̱fraꞌa̱ꞌ májánlaꞌ ríge̱ꞌ: Ana̱ꞌlóꞌ naꞌthán rí májáan a̱kui̱in mbáa xa̱bo̱ ga̱jma̱á majñu̱u̱ rí máján naꞌnii̱, ra̱ꞌkhá i̱ndo̱ó ga̱jma̱á majñu̱u̱ rí nanimbo̱o̱ꞌ. Xkuaꞌnii máꞌ ninimbo̱o̱ꞌ Raáb mangaa, a̱ꞌgo̱ tsí ni̱ngu̱jwa̱ mína̱ꞌ nákhí wajyúú. Ana̱ꞌlóꞌ niꞌthán rí májáan a̱kuii̱n ga̱jma̱á majñu̱u̱ rí máján niꞌnii̱: Ni̱grui̱gúu̱n ná goꞌwóo̱ xa̱bo̱ tsí nigó gúña̱maa xuajen Jerikó, jamí nimbáñúu̱ rí ma̱ga̱jnáa̱ ngu̱ꞌwa̱ ga̱jma̱á i̱mba̱ jamba̱a̱. Xómá rí mbáa xa̱bo̱ tsí nijáñuu, nánda̱a̱ xu̱u̱ꞌ, xkuaꞌnii máꞌ jaꞌnii rí nanimbo̱o̱ꞌ xa̱bo̱ mangaa, xí nda̱a̱ rí máján iꞌnii̱, nda̱a̱ mbá jayu numuu rí nanimbo̱o̱ꞌ. A̱ngui̱nꞌ, tsáanꞌ nanimba̱ꞌlaꞌ ju̱ya̱á Jesús, xánindxa̱ꞌ xúgiáanꞌlaꞌ xa̱bo̱ tsí nusngáá, numuu ndu̱ya̱ámálaꞌ rí i̱tha̱án gakhe̱ mi̱tra̱jwa̱ꞌ numa ikháanꞌxo̱ꞌ. Numuu rí mbá xúgiáanꞌlóꞌ na̱ngra̱ꞌáanꞌlóꞌ mbaꞌa nothon. Xí xtáa mbáa tsí na̱ꞌngo̱o̱ na̱ña̱wu̱u̱n rawuun ído̱ rí naꞌthán, xa̱bo̱ tsí nijráꞌáan máꞌ ñajuun tsúꞌkhue̱n, jamí na̱ꞌngo̱o̱ máꞌ rí naꞌthán ñajun mine̱e̱ꞌ mbá xúgíi̱ mangaa. Ído̱ rí nu̱xu̱ꞌdáaꞌlóꞌ xa̱ꞌ ñuu rawuun guáyo̱, na̱ꞌnga̱ꞌlóꞌ nu̱xu̱ꞌmáa̱ ma̱ꞌgee̱ ná nandalóꞌ rí ma̱ꞌgee̱, jamí na̱ꞌnga̱ꞌlóꞌ nu̱xma̱trígaa̱ mbá xúgíi̱. Gu̱ya̱xi̱ílaꞌ guꞌwá rguwa mba̱ꞌwo̱ rí na̱ka̱ ná inuu iya a̱pha̱ mangaa. Mbá ixe̱ lájwíin jayá ikhoo ne̱ rí na̱ka̱ ne̱ ná nandoo xa̱bo̱ tsí na̱ka̱ jayóo ne̱, maske máꞌ phú gakhe̱ irmajngua̱ꞌ ne̱ gíñá. Xúꞌkhue̱n máꞌ jaꞌnii rí ra̱ju̱u̱n xa̱bo̱ mangaa, mbá xuwi lájwíin ñajuun ne̱, jamí phú mba̱a̱ rí na̱ꞌngo̱o̱ ne̱ naꞌne. ¡Ra̱ꞌkhá tháán mba̱a̱ júba̱ ikha eꞌne mbá lájwíin ri̱ꞌyu̱u̱ agu rí nakhati̱yo̱o̱ꞌ! Rí ra̱ju̱u̱n xa̱bo̱ xómá ri̱ꞌyu̱u̱ agu jaꞌnii ne̱. Ka̱ma ne̱ ná xuyuu̱, jamí phú gíꞌdoo ne̱ tsiakhe̱ rí ma̱ꞌne ne̱ mbaꞌa inii rí ra̱máján, xúꞌkhue̱n máꞌ naꞌne maꞌchúu ne̱ mbá xúgíí xuyuu̱. Ndayá ski̱yu̱u̱ꞌ ne̱ ná nakha jámuu agu, jamí mbá xúgíí mbiꞌi naꞌne maꞌchúu ne̱ mbiꞌyuu̱. Tsí xa̱bo̱ numbaa na̱ꞌngo̱o̱ naꞌne másuu̱n jamí naxná ñajúu̱n mbá xúgíí inii xu̱kú xáná tsí namangu̱ún, xu̱kú xna, xu̱kú tsí nu̱xma̱tha mijná jamí xu̱kú tsí gatiin ná awúun iya a̱pha̱. Xómáꞌ rí ra̱ju̱u̱n xa̱bo̱ rá, nimbáa tsíꞌngo̱o̱ gáꞌthán ñajuun ne̱. Mbá rí xkawe̱ꞌ rí nda̱a̱ xó mu̱wajún thi̱ínlóꞌ ñajuun ne̱, jamí gajni̱í thana rí na̱gu̱díin xa̱bo̱ ná awúun ajngáa rí nagájnuu eꞌne ne̱. Ga̱jma̱á ra̱jwa̱nꞌlóꞌ nagájnuu ajngáa ná rawanlóꞌ rí mo̱ꞌne̱ mba̱a̱ Ana̱ꞌlóꞌ Mikhuíí, xúꞌkhue̱n máꞌ rí mu̱ꞌxná maꞌíinlóꞌ xa̱bo̱ tsí ni̱gu̱ma ku̱mii xómá jaꞌnii Ana̱ꞌlóꞌ. Mbóó máꞌ ná rawanlóꞌ nágájnuu ajngáa rí máján jamí ajngáa rí xkawe̱ꞌ. A̱ngui̱nꞌ, ra̱gíꞌmaa ma̱ꞌne rí xkuaꞌnii. Á ma̱goo ma̱ga̱jnúu iya ríná ná iduu iya rí thawuun dxe̱ꞌ. O ma̱goo maxná xndú rí mbiꞌyuu aseitúna̱ mbá ixu̱u̱ ígo̱ dxe̱ꞌ, o ígo̱ mbá ajmu̱u̱ úba̱ dxe̱ꞌ. Ma̱xáxóo a̱ngui̱nꞌ. Xkuaꞌnii máꞌ mangaa, ma̱xáxóo ma̱ga̱jnúu iya ríná ná iduu iya rí thawuun. Á xtáa mbáa tsí gakon rí ndaꞌyoo jamí nafroꞌo̱o̱ náá rí máján ma̱ꞌne ná majña̱ꞌlaꞌ dxe̱ꞌ. Gasngájmee̱ ne̱ ga̱jma̱á majñu̱u̱ rí gamakuii̱ jamí rí tsíkúxe̱ míne̱e̱ꞌ ído̱ rí naꞌnii̱ rí máján. Jamí xí i̱ndo̱ó máꞌ tsixígu̱ꞌ jamí sia̱nꞌ ríga̱ ná awúun a̱kia̱nꞌlaꞌ rá, mu̱xútha̱n tsiꞌyálaꞌ rí na̱ma̱ñalaꞌ jamí mu̱xúne̱ nduwalaꞌ rí no̱ne̱ nuwiinlaꞌ rí gakon. Numuu rí xa̱bo̱ tsí xkuaꞌnii i̱ma̱ñúú ra̱ꞌkhá ná inuu Ana̱ꞌlóꞌ i̱ꞌkha̱ rí na̱ma̱ñúu̱, rí ná numbaa ríge̱ꞌ i̱ꞌkha̱ ne̱, ná ju̱múu̱ máꞌ ikhii̱n jamí ná inuu gixa̱a̱. Ikhaa jngó, ná ríga̱ tsixígu̱ jamí sia̱nꞌ, ikhín máꞌ ríga̱ xkujndu jamí mbá xúgíí inii rí xkawe̱ꞌ mangaa. Xómáꞌ xa̱bo̱ tsí nduyáá jamí nafruꞌu̱ún rí na̱ꞌkha̱ ná inuu Ana̱ꞌlóꞌ rá, tsínii̱ rí xkawe̱ꞌ, tsítsa̱ñúu̱ꞌ gajmíi̱ xa̱bo̱, gamakuu̱n, májáan a̱kuíi̱n, phú nañáwíin a̱kuíi̱n juñúu̱ tsí ngíníi, gatíí rí máján nuni̱i̱, tsíraꞌwíi̱n xa̱bo̱ tsí mone̱ nga̱júu̱n jamí na̱nguá a̱jma̱ inúu̱. Tsí nandúún ma̱ri̱gá rí tsímáá ná numbaa ríge̱ꞌ, nu̱mba̱yíi̱ rí ma̱xátsa̱ñu̱ú xa̱bo̱. Asndo xó rí nudii̱ tsígoo rí máján jaꞌnii, kajngó mone̱ xa̱bo̱ rí nandoo Ana̱ꞌlóꞌ. Náá lá i̱ꞌkha̱ xkujndu jamí sia̱nꞌ rí ríga̱ ná majña̱ꞌlaꞌ rá. Ná awúun máꞌ a̱kia̱nꞌlaꞌ jamí ná awúun máꞌ rí phú na̱ni̱gua̱ꞌlaꞌ mu̱guaꞌdáálaꞌ. Phú na̱ni̱gua̱ꞌlaꞌ rí mu̱guaꞌdáálaꞌ rí ríga̱ ná tsu̱du̱u̱ numbaa jamí tsíguaꞌdáálaꞌ ne̱. Ra̱ꞌkhá tháán naxígua̱ꞌlaꞌ rí asndo nu̱radíinlaꞌ xa̱bo̱, ni máꞌ xúꞌkhue̱n tsíguaꞌdáálaꞌ rí nandalaꞌ. No̱ne̱ xkujndulaꞌ jamí na̱tsa̱ña̱ꞌlaꞌ. Ra̱kuáꞌdáálaꞌ numuu rí tsínda̱ꞌa̱álaꞌ Ana̱ꞌlóꞌ. Jamí ído̱ rí nu̱nda̱ꞌa̱laꞌ, tsídri̱gúlaꞌ rí nandalaꞌ numuu rí tsínda̱ꞌa̱laꞌ ne̱ rí ma̱jmaa ná ndaꞌyoo, rí nu̱nda̱ꞌa̱laꞌ ne̱ mu̱tsi̱jmálaꞌ ná rí na̱ni̱gua̱ꞌlaꞌ ikháanꞌlaꞌ. ¡Ra̱ꞌkhá xa̱bo̱ tsí nandúún juyáá i̱ndo̱ó Ana̱ꞌlóꞌ ñajwanlaꞌ! Á tsíya̱álaꞌ rí xa̱bo̱ tsí nandoo guéño jaꞌyoo rí ríga̱ ná numbaa, tsíyoo rí ma̱mbáxu̱u̱ꞌ ga̱jmáa̱ Ana̱ꞌlóꞌ rá dxe̱ꞌ. Ikhaa jngó, asndo tsáa máꞌ tsí nandoo guéño jaꞌyoo rí ríga̱ ná numbaa, tsímbáxu̱u̱ꞌ ga̱jmáa̱ Ana̱ꞌlóꞌ. O na̱ju̱ma̱ꞌlaꞌ rí nda̱a̱ numuu rí naꞌthán ná ajngáa rawuun Ana̱ꞌlóꞌ dxe̱ꞌ: \textless{}\textless{}Phú nandoo Ana̱ꞌlóꞌ jaꞌyoo Xe̱ꞌ rí kua̱ꞌa̱n ná xo̱xta̱ꞌlóꞌ, jamí mbáwíi̱ ikhaa̱ nandoo̱ rí ma̱galóꞌ juꞌyáa̱.\textgreater{}\textgreater{} Xómáꞌ ikhaa̱ phú mba̱a̱ rí máján naxnálóꞌ, rí ni ra̱jáꞌyalóꞌ. Ikhaa jngó naꞌthán ná ajngáa rawuun Ana̱ꞌlóꞌ: \textless{}\textless{}Ana̱ꞌlóꞌ na̱we̱je̱ thu̱ún xa̱bo̱ tsí nu̱xu̱xí mijná, xómáꞌ tsí nu̱xrígú mijná rá, naxnúu̱ rí máján, rí ni ra̱jáꞌñúu̱.\textgreater{}\textgreater{}Ikhaa numuu rúꞌkhue̱n, gu̱ni̱ꞌñá mijnálaꞌ rí maꞌthán ñajwanlaꞌ Ana̱ꞌlóꞌ. Ga̱wajún gúkuálaꞌ kajngó ma̱ꞌnga̱ꞌlaꞌ inuu gixa̱a̱, a̱ꞌkhue̱n rí ikhaa̱ magáyuu̱ ma̱ꞌgee̱ i̱mba̱ janíí. A̱guwalaꞌ ná inuu Ana̱ꞌlóꞌ, jamí ikhaa̱ maxuꞌma mine̱e̱ꞌ ná inalaꞌ. Xa̱bo̱ aꞌkhán, gu̱ni̱ꞌñáá ro̱ne̱laꞌ aꞌkhán. Tsáanꞌ a̱jma̱ a̱kia̱nꞌlaꞌ, i̱ndo̱ó máꞌ Ana̱ꞌlóꞌ gáju̱ma̱ꞌlaꞌ ju̱ya̱á. Gu̱ya̱álaꞌ rí mingínáaꞌlaꞌ, gu̱mbi̱ya̱ꞌlaꞌ jamí ga̱tájwíin a̱kia̱nꞌlaꞌ. Gu̱ni̱ꞌñáá ru̱ndu̱ꞌwa̱laꞌ, gu̱mbi̱ya̱ꞌlaꞌ. Gu̱ni̱ꞌñáá rajuwalaꞌ gagi, ga̱juwa jínálaꞌ. Gu̱xrígú mijnálaꞌ ná inuu Ana̱ꞌlóꞌ, jamí ikhaa̱ ma̱ꞌnii̱ rí phú ma̱gu̱ma mba̱ánꞌlaꞌ. A̱ngui̱nꞌ, nimbáa ma̱xáꞌthán tsu̱du̱u̱ nimbáa xa̱bo̱. Xí mbáa naꞌthán tsu̱du̱u̱ mbáa xa̱bo̱ o naꞌthée̱n rí ra̱máján iꞌnii̱ mbo̱ꞌ, naꞌthán tsu̱du̱u̱ xtángawoo Ana̱ꞌlóꞌ jamí naꞌthée̱n rí ra̱máján ne̱. Xí ikháán narathán rí ra̱máján xtángawoo Ana̱ꞌlóꞌ, tsíthane̱ mbáníí rí naꞌthán ne̱, rí nathane̱ mina̱ꞌ mbo̱ na̱ra̱jwa̱ꞌ numuu ne̱. I̱ndo̱ó máꞌ mbáwíi tsí nixná xtángoo ñajuun mbo̱ na̱ra̱jwa̱ꞌ, jamí i̱ndo̱ó máꞌ ikhaa̱ ma̱goo ma̱ꞌne jáwíi̱n o maxná maꞌíi̱n xa̱bo̱. Xómáꞌ ikháán rá, tsáa ñajwaanꞌ kajngó na̱tra̱jwa̱ꞌ numuu xa̱bo̱ ju̱ma̱a̱ꞌ róꞌ. Gu̱dxawíínlaꞌ ríge̱ꞌ, tsáanꞌ nu̱tha̱nlaꞌ: \textless{}\textless{}Xúge̱ꞌ o gátsíí mu̱ꞌgua̱lóꞌ xuajen rúꞌkhue̱n o ríge̱ꞌ, ma̱juwalóꞌ mbá tsigu ikhín, mu̱ngu̱jwa̱ ngaa̱lóꞌ jamí mu̱ꞌdaalóꞌ mbúkha̱a̱.\textgreater{}\textgreater{} Jamí nitsíya̱álaꞌ dí ga̱ri̱gá gátsíí, ni máꞌ tsíya̱álaꞌ xí xóó juwalaꞌ. Dílá ñajuun mbiꞌyalaꞌ rá. Xómá ru̱jmba̱aꞌ rí nakujma mbégo jamí i̱mbrúma nánda̱a̱ ne̱ xkuaꞌnii jaꞌñáaꞌlaꞌ. I̱wa̱á máján rí xáꞌnii gútha̱nlaꞌ: \textless{}\textless{}Xí Ana̱ꞌlóꞌ nandoo, ma̱juwalóꞌ jamí mo̱ꞌne̱lóꞌ ríge̱ꞌ o mo̱ꞌne̱lóꞌ rí ñúꞌún.\textgreater{}\textgreater{} Xómáꞌ ikháanꞌlaꞌ phú na̱ni̱gua̱ꞌlaꞌ rí mu̱tha̱n tsiꞌyálaꞌ, asndo xó rí ikháanꞌlaꞌ i̱tha̱n ñájwíín mbiꞌyalaꞌ jaꞌnii. Ra̱máján e̱ne̱laꞌ rí no̱ne̱ tsiꞌyálaꞌ xkuaꞌnii. Ikhaa jngó, tsí ndaꞌyoo máꞌ náá rí máján ma̱ꞌne jamí tsíꞌnii̱, xtáa̱ ná awúun aꞌkhán. Gu̱dxawíínlaꞌ ríge̱ꞌ, tsáanꞌ phú kuaꞌdáálaꞌ: gu̱mbi̱ya̱ꞌlaꞌ jamí gu̱ndxa̱ꞌwa̱ jínálaꞌ, numuu rí inu máꞌ ma̱ꞌkha̱ mbiꞌi rí mu̱míni̱ílaꞌ. Naꞌga máꞌ mbá xúgíí rí mitsaan kuaꞌdáálaꞌ, jamí naꞌpho̱ máꞌ ñu̱u̱ ruxi xtíñalaꞌ rí phú kuitsúun. Na̱ꞌkha̱a máꞌ iyoo mbúkha̱a̱ ajwa̱nꞌ mo̱jmo̱ꞌ jamí mbúkha̱a̱ ajwa̱nꞌ miꞌxá rí phú kuaꞌdáálaꞌ. Rúꞌkhue̱n maꞌthán rí ra̱máján ne̱ne̱laꞌ ído̱ gára̱jwa̱ꞌ Ana̱ꞌlóꞌ numalaꞌ, jamí matsikháanꞌlaꞌ ne̱. Ne̱ne̱ matíílaꞌ rí mu̱guaꞌdáálaꞌ awúun mbiꞌi rí inu máꞌ ma̱mbá numbaa ráanꞌ. Ikháanꞌlaꞌ túne̱ numi̱ilaꞌ xa̱bo̱ ngíníi tsí ni̱ñajun ná mbayalaꞌ. Gu̱dxawíínlaꞌ rí nandxaꞌwá raꞌa numa rúꞌkhue̱n ná inuu Ana̱ꞌlóꞌ tsí gíꞌdoo mbá xúgíí tsiakhe̱, jamí ikhaa̱ nidxawuu̱n máꞌ aꞋwúún yumbáá tsúꞌkhue̱n. Ikháanꞌlaꞌ phú nijuwa májánlaꞌ ná numbaa ríge̱ꞌ. Nitháan nda̱a̱ rí ndiꞌyálaꞌ, jamí ne̱ne̱laꞌ mbá xúgíí rí na̱ni̱gua̱ꞌlaꞌ. Niꞌngáanꞌlaꞌ me̱ndaꞌkho xómá xu̱kú tsí nitsíꞌyoo náá mbiꞌi ma̱ja̱ñúu̱. Ni̱rígulaꞌ ajngáa waꞌa tsu̱du̱ún tsí nda̱a̱ aꞌkhúún jamí ni̱radíi̱nlaꞌ, xómáꞌ ikhii̱n túxu̱da̱a̱ꞌ ñawúu̱n rí mu̱mba̱yú mi̱jne̱e̱. Ikhaa jngó, a̱ngui̱nꞌ, ga̱ꞌngo̱o̱ a̱kia̱nꞌlaꞌ asndo mbiꞌi rí ma̱ꞌkha̱a Táta̱ Jesukrísto̱. Gu̱ya̱xe̱laꞌ xómá eꞌne xa̱bo̱ tsí nañajun xanáá, gíꞌthu̱u̱n máꞌ xúꞌkhue̱n rí maguu̱ rí mitsaan nijma̱a̱ ná tsu̱du̱u̱ ju̱ba̱ꞌ rí niꞌdii̱, jamí gíꞌthu̱u̱n máꞌ xúꞌkhue̱n rí ma̱ga̱nú mbiꞌi rí ma̱ꞌkha̱a ruꞌwa. Xkuaꞌnii máꞌ góne̱ mangáanꞌlaꞌ, go̱ne̱ gakhe̱ a̱kia̱nꞌlaꞌ jamí ma̱xáꞌne níꞌnga̱ꞌlaꞌ rí mu̱waꞌthi̱ínlaꞌ, numuu rí inu máꞌ ma̱ꞌkha̱a Táta̱ Jesukrísto̱. A̱ngui̱nꞌ, nimbáa ma̱xáꞌne xkujndu ga̱jma̱á numuu nimbáa, kajngó ma̱xákujma aꞌkhánlaꞌ gáꞌne Ana̱ꞌlóꞌ. Gu̱ya̱álaꞌ rí inu máꞌ xtáa̱ ná rawuun xkrugua tsí ma̱ra̱jwa̱ꞌ. A̱ngui̱nꞌ, gu̱ya̱xi̱ílaꞌ xkri̱da xó rí nimíni̱í jamí xó rí nene̱ gakhe̱ a̱kui̱ín tsí ni̱rawí jnga̱a ajngóo Ana̱ꞌlóꞌ wajyúú. Ikháánlóꞌ na̱ju̱ma̱ꞌlóꞌ rí phú gagi júwa̱ꞌ tsí na̱ꞌngo̱o̱ a̱kui̱ín rí mu̱míni̱í. Ikháanꞌlaꞌ ni̱dxawíín xáne ni̱ꞌngo̱o̱ a̱kui̱in Jób rí mamínu̱u̱ꞌ me̱ndaꞌkho, jamí ndu̱ya̱á málaꞌ rí nda̱wa̱á phú mba̱a̱ rí máján nijanáa̱ neꞌne Ana̱ꞌlóꞌ. Numuu rí Ana̱ꞌlóꞌ phú nañáwíin a̱kuii̱n jamí phú mba̱a̱ a̱kuii̱n. A̱ngui̱nꞌ, nandoꞌ gátha̱nꞌlaꞌ i̱mba̱ rí gíꞌdoo numuu: ído̱ rí nu̱tha̱nlaꞌ numuu asndo dí máꞌ, xúxu̱ꞌdáaꞌlaꞌ Ana̱ꞌlóꞌ, ni máꞌ xútha̱nlaꞌ mbiꞌyuu asndo nimbá rí ríga̱ ná numbaa. Rí nu̱tha̱nlaꞌ rí mo̱ne̱laꞌ, ikhaa máꞌ góne̱laꞌ. Xí nu̱tha̱nlaꞌ \textless{}\textless{}ma̱ne̱\textgreater{}\textgreater{}, go̱ne̱laꞌ. Xí nu̱tha̱nlaꞌ \textless{}\textless{}ma̱xáne̱\textgreater{}\textgreater{}, xúne̱laꞌ. Kajngó ma̱xákujma aꞌkhánlaꞌ gáꞌne Ana̱ꞌlóꞌ. Xí xtáa mbáa tsí gíꞌdoo gaꞌkho̱ ná majña̱ꞌlaꞌ, gaꞌthán jáñuu̱ Ana̱ꞌlóꞌ. Xí xtáa̱ mbáa tsí nadxuu, ga̱ꞌsiee̱n ajmúú ná inuu Ana̱ꞌlóꞌ. Xí xtáa mbáa tsí najáñuu ná majña̱ꞌlaꞌ, gandxaꞌwúu̱n xa̱bo̱ buanuu tsí juya̱ idxu̱ún mbo̱ na̱gi̱mbíin. Kajngó muthán jáñíi̱ Ana̱ꞌlóꞌ ga̱jma̱á numuu̱, jamí mu̱tsua̱ꞌaa̱n aséite̱ ga̱jma̱á mbiꞌyuu Táta̱ Jesukrísto̱. Xí nuthan jáñíi̱ Ana̱ꞌlóꞌ ga̱jma̱á mbá jañii a̱kuíi̱n, ma̱ꞌni̱i a̱kui̱in tsí najáñuu, jamí Ana̱ꞌlóꞌ ma̱ꞌne rí ma̱tu̱xii̱. Jamí xí niꞌnii̱ aꞌkhán rá, Ana̱ꞌlóꞌ ma̱ꞌne mba̱a̱ a̱kui̱in jaꞌyoo̱. Ikhaa numuu rúꞌkhue̱n, mbámbáa gáꞌne maphú aꞌkhúun ná inuu i̱mba̱a̱, jamí mbámbáa gáꞌthán jáñuu Ana̱ꞌlóꞌ ga̱jma̱á numuu i̱mba̱a̱, kajngó ma̱ꞌnii̱ a̱kia̱nꞌlaꞌ. Xí mbáa xa̱bo̱ tsí májáan a̱kui̱in naꞌthán jáñuu Ana̱ꞌlóꞌ, phú gíꞌdoo tsiakhe̱ tsakuun rí naꞌnii̱. Ga̱rmáꞌáan a̱kia̱nꞌlaꞌ ju̱ya̱á Elía̱s, tsí ni̱gu̱wí jnga̱a wajyúú, ikhaa̱ ninindxu̱u̱ mbáa xa̱bo̱ numbaa xómá ikháánlóꞌ jayu. Nákhí niꞌthán jáñuu̱ Ana̱ꞌlóꞌ rí ma̱xáxnúu ruꞌwa, táxnúu ruꞌwa atsú tsigu i̱tikhu. Nda̱wa̱á a̱ꞌkhue̱n niꞌthán jáñuu̱ Ana̱ꞌlóꞌ mbu̱júu̱ rí maxnúu ruꞌwa, a̱ꞌkhue̱n nixnúu ruꞌwa neꞌne Ana̱ꞌlóꞌ, jamí nixnáa xndúu mbá xúgíí inii rí ndaꞌya ná inuu ju̱ba̱ꞌ. A̱ngui̱nꞌ, tsáanꞌ nanimba̱ꞌlaꞌ ju̱ya̱á Jesús, xí mbáa tsí ikháanꞌlaꞌ naniñuu jamba̱a̱ rí gakon, jamí i̱mba̱a̱ naꞌne rí ma̱tangaa̱, gu̱ya̱álaꞌ rí tsí nambáyúu xa̱bo̱ aꞌkhán rí ma̱tanga̱a a̱kui̱in jamí maniñuu raꞌne aꞌkhán, naꞌnii̱ rí ma̱jríya̱a̱ꞌ rí ma̱ja̱ñúu̱ jamí naꞌnii̱ rí Ana̱ꞌlóꞌ ma̱ꞌne mba̱a̱ a̱kui̱in jaꞌyoo mbaꞌa aꞌkhán rí niꞌnii̱.}
\end{description}
{\needspace{3\baselineskip}
\vspace{15pt}\noindent
\fontsize{13}{15.6}\selectfont \textbf{{\noindent
\raisebox{\baselineskip}[0pt]{\pdfbookmark[2]{{C.2 } Chinantec Full Text}{s-cso}}\raisebox{\baselineskip}[0pt]{\protect\hypertarget{s-cso}{}}{C.2 }Chinantec Full Text}}\markright{Chinantec Full Text}
\XLingPaperaddtocontents{s-cso}}\par{}
\penalty10000\vspace{10pt}\penalty10000\needspace{5\baselineskip}

\penalty-3000
\begin{description}
\setlength{\topsep}{0pt}\setlength{\partopsep}{0pt}\setlength{\itemsep}{0pt}\setlength{\parsep}{0pt}\setlength{\parskip}{0pt}\setlength{\leftmargini}{1em}\setlength{\leftmarginii}{1em}\setlength{\leftmarginiii}{1em}\setlength{\leftmarginiv}{1em}\penalty10000\item[ISO 639-3 code of language:]{\textsquarebracketleft{}cso\textsquarebracketright{}}
\penalty10000\item[Title of the text:]{Sí² Quioh²¹ Santiago. JÚ¹ CHÚ³² QUIOH²¹ JESÚS TSÁ² LƗŃ³ CRISTO (El Nuevo Testamento en el chinanteco de Sochiapan)}
\needspace{5\baselineskip}

\penalty-3000\item[Cited as:]{La Liga Bíblica. 2009. Sí² Quioh²¹ Santiago. JÚ¹ CHÚ³² QUIOH²¹ JESÚS TSÁ² LƗŃ³ CRISTO (El Nuevo Testamento en el chinanteco de Sochiapan), 525-33. La Liga Bíblica. \textless{}Accessed: 12 June 2012\textgreater{}. \href{http://www.scriptureearth.org/data/cso/PDF/00-WNTcso-web.pdf}{\textcolor[rgb]{0,0,0}{http://www.scriptureearth.org/data/cso/PDF/00-WNTcso-web.pdf}}}
\needspace{5\baselineskip}

\penalty-3000\item[Text provenance:]{The actual text used and processed was the SFM file received from Chinanteco de Sochiapan language development and Bible translation team. This text is included in the publicly available work as indicated in the work is cited. Date of acquisition of the texts from the team was: 13. June 2011. \_\_(do I need to take out the *? do I need to take out the double spaces?)\_\_}
\needspace{5\baselineskip}

\penalty-3000\item[What I did to the text before using it in comparisons:]{Section headers were removed. Chapter and verse numbers were removed.}
\penalty10000\item[Copyright holder as indicated:]{SIL International and the language development team. Used by written permission.}
\penalty10000\item[The Text:]{Jná¹³ la³² Jacobo, tsá² lı̵́n³ jná¹³ jan² *tsá² má²dí¹hlánh¹ joh¹ Dió³² jɨ³ Jesucristo Tɨ³² Juo¹³ dí², juanh³² jná¹³: “Hia² hnoh²”, tá¹ quia³tún³ nió³ hnoh² tsá² *Israel, tsá² má²ná¹yanh³² náh² tá¹ cáun² hngá¹máh³. Hnoh² reh², ma³hiún¹³ hnoh² honh² lɨ³ua³ cáun² hi³ quiunh³² náh², quí¹ la³ cun³ hi³ má²ca³lɨ³ ñíh¹ hnoh² jáun² hi³ tɨ³ jlánh¹ bíh¹ re² lı̵́²tɨn² tsú² hi³ jmu³ juenh² tsı̵́³, nı̵́¹juáh³ zia³² hi³ cá² lau²³ ca³tɨ²¹ hi³ taunh³² tsú² jáun² ta²¹. Hi³ jáun² né³, chá¹ hnoh² cáun² honh², hi³ jáun² lı̵́¹³ lɨ³tɨn² hnoh² re² hi³ jmúh¹³ náh² juenh² honh², hi³ jáun² hnoh² lı̵́¹³ lı̵́n³ náh² tsá² má²hún¹ tsı̵́³, tsá² má²ca³hiá² ca³táunh³ ca³la³ tán¹ hián² cu³tí³, la³ cun³ tsá² tiá² hi³ lɨ³hniauh²³ hí¹ cáun² ñí¹con² yáh³. Lɨ³ua³ jan² hnoh² tsá² tiá² re² má²jniá³ jmı̵́¹ honh² náh², mı̵́¹ náh² ñí¹con² Dió³², hi³ jáun² lı̵́¹³ hián¹³ náh², quí¹ hí³ bíh¹ cue³² ca³la³ hi³ lı̵́¹ má²tú² má²ziáun²³ ñí¹con² ca³la³ jı̵́n³² tsáu², ha³ tiá² jin²³ yáh³ tsú² tsá² hiú² tsá² mɨ³² ñí¹con². Tɨ³la³ hniáuh³² mı̵́³² tsú² hi³ hu²¹ cáun² tsı̵́³ má¹ná¹, hí¹ cú¹pih²¹ yáh³ tiá² hniáuh³² hu²¹ tun³ tsı̵́³ tsú²; quí¹ nı̵́¹juáh³ hi³ hu²¹ tun³ tsı̵́³ tsú², jáun² lı̵́n³ tsú² la³jmı̵́¹ lı̵́³ cu³ jláɨ³ jmı̵́²miih²¹ bíh¹, hi³ hlia³² chí³ tɨ³ hlá² tɨ³ nı̵́². Tsá² la³ hí³ tiá² hniáuh³² yáh³ hi³ cáun² lı̵́¹ lı̵́n¹³ hi³ hiáuh³ hi³ jmı̵́¹ cué²¹ jáun² Dió³² Juo¹³ dí²; quí¹ tsá² la³ hí³ dá² cáun² lı̵́¹ tı̵́² lı̵́¹ jeinh³² tsı̵́³ tɨ³ hlá² tɨ³ nı̵́² bíh¹. Hi³ jáun² né³, cuı̵́¹ jmu¹ tsah³ tsá² reh² dí², tsá² tsı̵́¹juı̵́³, quí¹ cun³ñí¹ hi³ má²ca³ta³zanh¹ tsú² re². Hi³ tsá² hánh³ né³, cuı̵́¹ jmu¹ tsah³ quí¹ cun³ñí¹ hi³hliá² má²ca³méih³ hi³ quien² tsú², quí¹ tsá² ná¹hánh³ dá² lı̵́n³ la³jmı̵́¹ lı̵́³ lí¹³ hi³ tiá² má²hı̵́e² bíh¹. Quí¹ nı̵́¹ má¹ca³hiá² hiú², hi³ má¹lɨ³² jáun² né³, chei³² lı̵́n³², jáun² lı̵́²quiéin² bíh¹ náɨ², hi³ suh³² siáh³ lí¹³ quioh²¹, jáun² lı̵́¹ ca³há² ca³nga² bíh¹ hi³ jmı̵́¹ jniá³ jáun² chú³² lı̵́n³². La³ jáun² bíh¹ quiunh³² uá²jaɨ³² tsá² hánh³, cu³diá²jan² bíh¹ tsá²hín³ tsú² tá¹la³ cuá¹jmu² jáun² tá¹³. Hen³ ma³ jmı̵́¹ ren² tsá² tion² tsı̵́³ ca³la³ jı̵́³² hi³ quiunh³²; quí¹ nı̵́¹ má¹ca³nga² tsú² la³jı̵́³² hi³ jáun², hiáuh³ bíh¹ tsú² la³ cun³ hi³ ca³jmú³ jáun² Dió³² jáɨ¹³ hi³ cué³² ñí¹con² tsá² hniau³ dí², hi³ lı̵́¹³ zian² dí² quionh³ tsá² ca³cah¹³ hí³ jmáɨ¹ quionh³. Tɨ³la³ nı̵́¹juáh³ ca³chó³² cáun² jmáɨ¹ hi³ hniá¹ tsú² cáun² hi³ tiá² ca³tɨ²¹ né³, tiá² hniáuh³² yáh³ hi³ cáun² lı̵́¹ lı̵́n¹³ tsú² hi³ Dió³² bíh¹ cá² lau²³ cun³quionh³ hi³ hlah³ jáun². Quí¹ hí¹ cónh³ yáh³ tiá² lı̵́¹³ hi³ hen² tsı̵́³ Dió³² hi³ hlah³, sa³jun³ cá² lan²³ tsú² tsáu² cun³quionh³ hi³ hlah³. Tɨ³la³ nı̵́¹juáh³ hi³ hniá¹ tsú² la³ cun³ hi³ tiá² ca³tɨ²¹ né³, jáun² hi³ jáun² má²ja³² ca³la³ tɨ³ ñeh² hñu³ tsı̵́³ hnga² bíh¹ tsú²; quí¹ hñu³ tsı̵́³ hnga² bíh¹ tsú² dí¹quian¹ hi³ jáun². Hi³ nı̵́¹ má¹ca³lı̵́n¹³ tsú² jmu³ la³ cun³ hi³ má²hniá¹ jáun² jmu³ né³, má¹jáun² má²lı̵́¹³ tso³ bíh¹. Jáun² nı̵́¹ má¹ca³lɨ³ pin³ tso³ tán¹ hián² cu³tí¹³ né³, jáun² má¹jáun² má²jún³ bíh¹ tsú² ñí¹con² Dió³². Hnoh² reh², tsá² hnau³ jná¹³ lı̵́n³², ha³ lı̵́² lı̵́¹ cuı̵́¹ lɨ³can² hnoh²; quí¹ ca³la³ jɨ³lɨ³² hi³ chu²¹, jɨ³ la³jı̵́³² hi³ quien² ca³la³ tán¹ hián², cáun² lı̵́¹ cue³² bíh¹ Dió³² Jméi² dí², Tsá² cuá³ hñu³mɨ³cuú². Hí³ bíh¹ Tsá² lı̵́n³ mí¹zioh²¹ quioh²¹ la³jı̵́³² joh¹ hi³ jniá³ jáun² tɨ³ chi³cuú²; Dió³² Tsá² lı̵́n³ la³ má²lı̵́n³ jáun² yáh³. Tɨ³la³ Tsá² hí³ má¹ná¹, tiá² zan² yáh³ la³jmı̵́¹ za² joh¹ hi³ ja³² jáun² chi³cuú², hi³ cáun² lı̵́¹ jmu² hính¹³ tɨ³ hlá² tɨ³ nı̵́². Hí³ bíh¹ Tsá² ca³lı̵́n¹³ hi³ ma³jniau² dí² hi³ hmaɨ²¹ cun³quionh³ jú¹ tson² quioh²¹. Cun³ñí¹ hi³ jáun² né³, má²láɨ³ dí² tsá² má²ná¹chan¹ ñí¹con² tsú² ja¹ quioh²¹ la³jı̵́³² hi³ ca³jmú³ tsú² jáun², la³jmı̵́¹ má²ná¹chan¹ jáun² la³jı̵́³² hi³ lau³² la³ñí¹. Hi³ jáun² né³ reh², tsá² jlánh¹ hnau³ jná¹³, hniáuh³² lɨ³ ñíh¹³ náh² hi³ tianh¹³ náh² hua³jan²¹ hi³ náɨh¹ náh² jáɨ¹³ hi³ hleh³² tsáu²; tɨ³la³ tiá² hniáuh³² yáh³ hi³ la³juɨ³² hléh¹³ hnoh² má¹ná¹, sa³jun³ hniáuh³² hi³ la³juɨ³² má²míh³ honh² náh², quí¹ tsá² míh²³ tsı̵́³ cun³ tiá² lı̵́¹³ bíh¹ jmu³ la³ cun³ hi³ tɨ³² tsı̵́³ Dió³². Hi³ jáun² bíh¹ hniáuh³² cháh¹³ hnoh² tɨ³ có³² la³jı̵́³² hi³ tiá² jɨ² hi³ má²chí¹mɨ³² jáun² honh² hnoh², la³ má²quionh³ jɨ³lɨ³² hi³ hlah³ hi³ jlánh¹ má²ca³lɨ³ hliáun³ jáun²; tɨ³la³ hi³ nio² re² honh² bíh¹ hnoh² cuéh¹ náh² jáɨ¹³ hi³ lı̵́¹³ lɨ³ quien² jú¹ tson² hi³ má²ca³jñí³ jáun² Dió³² hñu³ honh² náh², la³ cun³ jáɨ¹³ hi³ jmu³ hi³ liáun³ náh². Hi³ jáun² hniáuh³² má¹tih²¹ bíh¹ hnoh² la³ cun³ hi³ hɨe³² jú¹ tson² jáun², quí¹ nı̵́¹juáh³ cáun² ti³ lı̵́¹ naɨh³² náh², jáun² la³ jáun² lı̵́¹ lı̵́²can² náh² hmóu³² bíh¹. Quí¹ tsá² ti³ lı̵́¹ náɨ³² jú¹ tson² hi³ tiá² má²tí³² yáh³ tsú² má¹lɨ³² jáun², jáun² tsá² la³ hí³ dá² lı̵́n³ la³jmı̵́¹ lı̵́n³ jan² tsá² má²lı̵́²jı̵́e³ ñí¹ cun³quionh³ cáun² hu³² bíh¹. Quí¹ nı̵́¹ má¹lı̵́²jı̵́e³ tsú² lánh³ lı̵́³ ñí¹, jáun² la³juɨ³² tsá²có³ bíh¹ tsı̵́³ tsú² lánh³ lı̵́³ ñí¹ nı̵́¹ má¹lɨ³² jáun². Tɨ³la³ tsá² taun² re² ñí¹ liei²¹ chu²¹ quioh²¹ Dió³² né³, liei²¹ hi³ jmu² jáun² hi³ lı̵́¹³ liáun³ tsú², hi³ má²tí³² tsú² re², hi³ tiá² tsá²có³ tsı̵́³ tsú² hi³ náɨ³², jáun² tsá² la³ hí³ bíh¹ jlánh¹ lɨ³ hiún² tsı̵́³ ca³tɨ²¹ la³ cun³ la³jı̵́³² hi³ jmu² jáun². Nı̵́¹juáh³ zian² tsá² lı̵́n²³ hi³ jlánh¹ re² má²tí³² cuáh³² quioh²¹, tɨ³la³ nı̵́¹juáh³ cáun² lı̵́¹ ja³tú² ja³lı̵́²³ bíh¹ ho³ tsú² má¹lɨ³² jáun², jáun² tsá² la³ hí³ cáun² lı̵́¹ má²can² hnga² bíh¹, hi³ tiá² lin¹ lı̵́²quien² yáh³ hi³ jlánh¹ má²tí³² tsú² jáun² cuáh³². Tɨ³la³ tsá² má²tí³² cuáh³², cáun² hi³ jɨ² hi³ hngó³² ta³ ñí¹ Dió³² Jméi² dí² né³, jáun² tsá² la³ hí³ bíh¹ jmu² la³ lá²: Má²hon³ tsú² tsá¹míh¹ hnáɨ³, jɨ³ tsá²mɨ³ hnáɨ³ nı̵́² siáh³ ja¹ lɨ³ua³ cáun² uu³mí²tsɨ³² hi³ zian² tsú², hi³ hua³hín¹³ tsú² siáh³ hnga² hi³ tiá² ma³quiá¹ jmı̵́²chí³ quioh²¹ cun³quionh³ hi³ zia³² jáun² ñí¹ hué²¹ lá². Hnoh² reh², tsá² tanh²³ náh² ta²¹ Jesucristo Tɨ³² Juo¹³ dí², Tsá² cú² juenh², tiá² ca³tɨn¹ ya³ náh² jmúh¹³ quien² jan², hi³ jan² tiá² quien². Jmú¹³ jáɨ¹³: Nı̵́¹juáh³ ja¹ ñí¹ ná¹ngɨh³² náh² ca³hi³ jan² tsá² quih³² hmɨh³² chu²¹, hi³ tioh³ siáh³ anillo hi³ lı̵́³ cú¹mí¹niau²¹ cuo² tsú², hi³ tá¹la³ jáun² né³ ca³hi³ siáh³ jan² tsá² tia³mii² quih³² tsı̵́¹ hmɨh³² tseh¹, tɨ³la³ nı̵́¹juáh³ hnoh² jlánh¹ ca³ma³quien¹³ náh² tsá² quih³² hí³ hmɨh³² chu²¹, hi³ juoh¹ náh² tsú² la³ lá²: “Ñí³ ñí¹ chu²¹ lá²”; tɨ³la³ tsá² tia³mii² hí³ né³ juoh¹ náh² la³ lá²: “Tɨ³ ó³² cuá²náu² hnú², ho³lá²dá² hué³² lɨn²¹ lá² ñí³”, hi³ nı̵́¹juáh³ la³ jáun² bíh¹ la³ jmuh³² náh² né³, jáun² má²ná¹jmuh³² náh² quien² jan², hi³ jan² tiá² quien² bíh¹ ja¹ quián¹³ náh² hmóu³², hi³ má²ná¹chú¹ ná¹ñih²¹ náh² tsáu² hi³ hu²¹ hlah³ honh² náh². Hnoh² reh², tsá² jlánh¹ hnau³ jná¹³ lı̵́n³², niéi² náh² re² jáɨ¹³ hi³ juanh²¹ jná¹³ lá²: Dió³² má²ca³quianh³ tsá² tia³mii² zian² ñí¹ hué²¹ lá², hi³ jáun² cun³quionh³ hi³ nio² cáun² tsı̵́³ tsú² ñí¹con² Dió³², lı̵́¹³ lı̵́n³ tsú² jan² tsá² hánh³, ¿tiá¹ tson²? Tsá² la³ hí³ bíh¹ hiáuh³ jáɨ¹³ hi³ tsa³tiánh² ñí¹ cuá¹quien² Dió³², la³ cun³ jáɨ¹³ hi³ ca³jmú³ jáun² Dió³² ñí¹con² tsá² cu³tí³ hniau³ hí³ dí². Tɨ³la³ hnoh² má¹ná¹, chin²³ náh² hua³heih²¹ tsá² tia³mii². ¿Haun¹³ juáh¹³ chín¹dí¹ tsá² hánh³ bíh¹ tsá² jmu² hian² tsı̵́³ hi³ quin²³ hi³ quián¹³ hnoh² dúh¹, hi³ tú² hna² tsú² siáh³ ñí¹ta²¹? Hnoh², tsá² má²ná¹láɨn¹ náh² Cristo, ¿haun¹³ juáh¹³ chín¹dí¹ tsá² hánh³ hí³ bíh¹ tú² hión³² hi³ jmáɨ² tsú² dúh¹, hi³ jlánh¹ bíh¹ chu²¹ jáun², hi³ má²ná¹jmáɨ² náh² jáun² ta³né³²? Hi³ jáun² né³, nı̵́¹juáh³ cu³tí¹³ ma³tih²¹ náh² liei²¹ hi³ quien² jáun² la³ñí¹, la³ cun³ jáɨ¹³ hi³ rá¹juáh³ jáun² ñí¹ Sí² quioh²¹ Dió³², hi³ juáh³ jáun² la³ lá²: “Hniáuh³² má¹hno¹ hnoh² tsá²mɨ³cuóun² renh² la³jmı̵́¹ hno³ náh² hmóu³²”, jáun² má²ná¹jmuh³² re² bíh¹ hnoh² nı̵́¹juáh³ la³ jáun². Tɨ³la³ nı̵́¹juáh³ jmuh³² náh² quien² jan² né³, hi³ jan² tiá² quien², jáun² tso³ bíh¹ má²ná¹jmuh³² náh², quí¹ liei²¹ quioh²¹ Dió³² jmu² lin¹ hi³ tiá² jmuh³² yáh³ hnoh² la³ cun³ hi³ ca³tɨ²¹. Quí¹ nı̵́¹juáh³ jan² tsá² má²tí³² la³jı̵́³² héih³² quioh²¹ liei²¹, tɨ³la³ sa³ la³ zia³² bíh¹ cáun² hi³ tiá² ca³ma³tí³² tsú² má¹lɨ³² jáun², jáun² má²ren² bíh¹ tsú² tso³, la³juah²¹ dúh¹ hi³ tiá² lı̵́² ma³tí³² hí¹ cáun² bíh¹ tsú² héih³² quioh²¹ liei²¹ lı̵́²³. Quí¹ Dió³² ca³juáh³ la³ lá²: “Ha³ lı̵́² jmú² náh² juón¹”, hi³ ca³juáh³ siáh³: “Ha³ lı̵́² jngɨh³ náh² tsáu²”. Jáun² né³, uá¹jinh¹ jan² tsá² tiá² lı̵́¹ jmú² lı̵́¹ tsau³², tɨ³la³ já² jngɨh² bíh¹ tsú² tsáu², jáun² ñí¹ nı̵́² bíh¹ tsá²hín³ tsú², hi³ tiá² la³ lı̵́²ma³tí³² yáh³ tsú² héih³² quioh²¹ liei²¹ jáun². Hniáuh³² hléh¹³ hnoh² jɨ³ hi³ má¹zián¹³ náh² la³jmı̵́¹ zian² tsá² ná¹ñi³² hi³ chau²¹ jmáɨ¹ tá¹tsɨ²¹ héih³² ñí¹con² bíh¹, la³ cun³ lı̵́³ jáun² héih³² quioh²¹ Cristo hi³ jmu² hi³ lı̵́¹³ liáu³ dí². Quí¹ tsá² tiá² jmu² juenh² tsı̵́³ ñí¹con² tsáu², sa³jun³ jmu³ yáh³ Dió³² juenh² tsı̵́³ ñí¹con² tsú² hnga² nı̵́¹ má¹cá²chó³² jmáɨ¹ hi³ ra³tsa² Dió³² héih³² quioh²¹ tsú². Tɨ³la³ nı̵́¹juáh³ tsá² jmu² juenh² tsı̵́³ ñí¹con² tsáu² né³, jáun² tiú²uú² zia³² yáh³ hi³ háɨnh¹³ ñí¹con² tsú² nı̵́¹ má¹ca³ta³tsɨn¹ tsú² héih³². Hnoh² reh² ¿ha³ dá² jinh¹ quien² hi³ juáh³ tsú² hi³ nio² cáun² tsı̵́³ ñí¹con² Dió³², nı̵́¹juáh³ tiá² lin¹ hi³ chu²¹ jmu² yáh³ tsú²? ¡Cun³ tiá² lı̵́¹³ bíh¹ liáu³² hi³ nio² jáun² cáun² tsı̵́³ tsú²! Jmú¹³ jáɨ¹³: Nı̵́¹juáh³ zian² jan² tsá²mɨ³ reh² dí², ho³lá²dá² tsá²ñuh² hi³ tiá² ha¹ zia³², sa³jun³ zia³² bíh¹ hi³ cuh³ tsú² cun³jmá² cun³ jnia³², hi³ má¹lɨ³² jáun² né³, ma³ jan² hnoh² ti³ lı̵́¹ juouh³² náh² tsá² hí³ la³ lá²: “Cuánh² cáun² hi³ re² hi³ tɨn²; cuı̵́¹ lɨ³chanh³² nú², hi³ cuı̵́¹ conh²¹ siáh³ henh¹ nú²”, tɨ³la³ já² hí¹ cáun² yáh³ tiá² hi³ lı̵́²cuéh³ náh² tsú² má¹lɨ³² jáun², ¿ha³ dá² jinh¹ quien² hi³ juáh³ hnoh² la³ jáun²? La³ jáun² bíh¹ lı̵́³ hi³ tiá² lin¹ zia³² ñí¹ jmu³ ta²¹ yáh³ hi³ ti³ lı̵́¹ juáh²³ tsú² hi³ nio² cáun² tsı̵́³ ñí¹con² Dió³², nı̵́¹juáh³ tiá² lin¹ zia³² yáh³ hi³ jmu² tsú². Hi³ lı̵́¹ juáh²³ tsú² jáun² hi³ nio² cáun² tsı̵́³, jáun² má²lı̵́³ la³jmı̵́¹ lı̵́³ cáun² hi³ jún¹ bíh¹ jaun³². Jáun² né³, lı̵́¹³ zian² tsá² záɨh³² raɨnh²¹ la³ lá²: “Hnú² bíh¹ jan² tsá² nio² cáun² honh², tɨ³la³ jná¹³ né³, lı̵́n³ ná¹ jan² tsá² jmu² ta²¹ bíh¹”. Tɨ³la³ jná¹³ né³ juanh³²: Ma³li²¹ hnú² ha³ lánh³ rón³² nio² cáun² honh² nı̵́¹juáh³ hnú² jan² tsá² tiá² lin¹ zia³² hi³ jmuh³²; hi³ jáun² jná¹³ má¹li²¹ hi³ nio² jáun² cáun² tsɨn³² cun³quionh³ hi³ jmu² ná¹. ¿Hí¹ má²nio² cáun² honh² hnú² hi³ zian² jan² tán¹ Dió³²? Chú³² bíh¹ nı̵́¹juáh³ lı̵́n²³ hnú² la³ jáun². Ja³bí¹ jmı̵́²chí³ hláɨnh¹ yáh³ ná¹lı̵́n²³ la³ jáun², hi³ ca³la³ quí² juan²³ bíh¹ hi³ juénh². Hnú², ¡tsá² cáun¹ nú²! Sá¹nı̵́² cónh³ jmı̵́¹ jmu¹ jná¹³ lin¹ hi³ tiá² lin¹ ñí¹ quien² hi³ juáh²³ tsú² hi³ nio² cáun² tsı̵́³, nı̵́¹juáh³ tiá² lin¹ zia³² yáh³ hi³ chu²¹ hi³ jmu² tsú². Jáun² né³, ¿haun¹³ juáh¹³ hnga² hla¹ Há²bran²¹, jméi² dí¹hio³ ñú¹deh³ dí², ca³lɨn³ la³juah²¹ dúh¹ jan² tsá² chun¹ ñí¹con² Dió³² cun³quionh³ hi³ ca³jmú³ tsú² jáun², jmı̵́¹ ca³jéin³² tsú² jáun² Isaac jon² ñí¹con² Dió³², hi³ ca³ra³tsɨn² tsú² jáun² ñí¹hiú¹³ míh¹? Jáun² né³, cun³quionh³ la³ nı̵́² ca³lɨ³ lin¹ hi³ nio² cáun² tsı̵́³ tsú², ¿tiá¹ tson²? Hi³ jáun² bíh¹ ca³hia³ ca³táuh³ tán¹ hián² cu³tí³ hi³ nio² jáun² cáun² tsı̵́³ tsú² cun³ñí¹ hi³ zia³² hi³ ca³jmú³ tsú² jáun². La³ jáun² bíh¹ ca³lɨ³tí³ jáɨ¹³ quioh²¹ Dió³² ñí¹ rá¹juáh³ jáun² la³ lá²: “Dió³² ca³heh³ Há²bran²¹ la³juah²¹ dúh¹ jan² tsá² chun¹ cu³tí³ cun³ñí¹ hi³ ca³chá³ tsú² jáun² cáun² tsı̵́³”. La³ jáun² bíh¹ lɨ³ hi³ ca³lɨn³ tsú² há²mei²¹ joh¹ Dió³². Jáun² né³, cun³quionh³ la³ nı̵́² má²ca³lɨ³ lin¹ hi³ jun³juáh¹³ cun³quionh³ jmáh³la³ hi³ nio² cáun² tsı̵́³ yáh³ tsú² lı̵́¹³ lı̵́n³ tsú² la³juah²¹ dúh¹ jan² tsá² chun¹ ñí¹con² Dió³², tɨ³la³ hniáuh³² hi³ zia³² hi³ jmu² bíh¹ tsú² uá²jaɨ³². Ja³bí¹ la³ jáun² ca³lɨ³ siáh³ ñí¹con² hla¹ tsá²mɨ³ Rahab, tsá² lı̵́¹ jmı̵́¹ jmú² jmı̵́¹ tsau³² hí³ hi³ quí² jéin³² hnga². Dió³² ca³heh³ tsá² hí³ la³juah²¹ dúh¹ jan² tsá² chun¹ cu³tí³ jmı̵́¹ ca³cué³ tsú² jáun² hñú¹³ ñí¹con² tsá² ca³ñí¹quí¹ ñí¹lian¹³ hí³ juú²co¹ tsú², tsá² zéin¹ hí³ hla¹ Josué, hi³ ca³hı̵́e³ tsú² siáh³ cáun² juɨ³² siáh³ hi³ tsa³tánh¹ tsú² jmı̵́¹ tɨ³ ñí¹ tsa³tánh¹. Jáun² né³, tiá² lin¹ ñí¹ jmu² ta²¹ yáh³ hi³ ti³ lı̵́¹ juáh³ tsú² hi³ nio² cáun² tsı̵́³, nı̵́¹juáh³ tiá² lin¹ zia³² yáh³ hi³ jmu² tsú²; la³jmı̵́¹ tiá² ta²¹ jmu² ngú³ nı̵́¹juáh³ hi³ tiá² jmı̵́²chí³ hu²¹. Hnoh² reh², tiá² hniáuh³² hi³ zian² juóun³² tsá² jmu² pí³ hi³ lı̵́n¹³ tɨ³² ja¹ quián¹³ hnoh², quí¹ la³ cun³ hi³ má²né¹ dí² hi³ tɨ³ jlánh¹ bíh¹ huáh² tsı̵́³ héih³² hi³ tá¹tsɨn¹ tsá² ná¹lı̵́n³ tɨ³². ¿Haun¹³ juáh¹³ la³jáɨ³² bíh¹ dí² jmu³² hi³ tiá² ca³tɨ²¹ dúh¹? Tɨ³la³ nı̵́¹juáh³ zian² jan² tsá² tiá² hleh³² hí¹ cu³ jéin³² cáun² hi³ tiá² ca³tɨ²¹ hléh³², jáun² tsá² la³ hí³ lı̵́n³ jan² tsá² má²ca³hiá² ca³táunh³ tán¹ hián² cu³tí³ bíh¹, jan² tsá² má²tɨn² jmu² re² lı̵́n³² héih³² ñí¹con² hnga². Quí¹ nı̵́¹ má¹ca³táunh¹³ dí² mí¹ñí² ho³ tsa³cuá¹, jáun² taunh¹³ jáh³ ta²¹; má¹jáun² né³, má²lı̵́¹³ bíh¹ jmú¹³ dí² héih³² ñí¹con² jáh³. Ja³bí¹ la³ jáun² lı̵́³ siáh³ quioh²¹ mu² cáh¹. Uá¹jinh¹ cáh¹ lı̵́n²¹, hi³ pin³ lı̵́n³² siáh³ chí³ hi³ hlia³², tɨ³la³ cun³ jáun² tsá² cuá¹quian³² mu², tióh³² bíh¹ tsú² jmu² héih³² ñí¹con² mu² jáun² hi³ tsó³² juɨ³² ñí¹ hnió³ tsú² hi³ tsó³² cun³quionh³ zı̵́h¹ mu² hi³ lı̵́³ jáun² cun³quionh³ cáun² jo²¹ hmá² míh¹ hi³ he² jáun² tɨ³ cu³hna²¹ hi³ hu²¹ jáun² chu³ jmáɨ². La³ jáun² bíh¹ lı̵́³ siáh³ zı̵́h¹ dí² uá²jaɨ³², uá¹jinh¹ cáun² hi³ pih²¹ lı̵́n²¹ bíh¹ jaun³², tɨ³la³ cun³ jáun² hliáun³ lı̵́n³² bíh¹ hi³ jmu². Uá¹la³ cun³ cáun² sı̵́² pih²¹, cun³quionh³ hi³ pih²¹ jáun² bíh¹ lı̵́¹³ có³² cáun² já¹hngá¹ pa²¹ lı̵́n²¹. Hi³ ja³bí¹ zı̵́h¹ dí² siáh³ lı̵́³ la³jmı̵́¹ lı̵́³ cáun² sı̵́². Zı̵́h¹ dí² jáun² bíh¹ jlánh¹ hlah³ la³ cónh³ bíh¹ la³jı̵́³² ñí¹ pih²¹ ñí¹ siún¹ quiú¹³ dí², hi³ jáun² tá¹ jan² bíh¹ dí² má²tsá²hliánh² lı̵́²³. Sı̵́² hi³ chí¹hún¹ jáun² zı̵́h¹ dí² ja³² la³ tɨ³ quiu³juóu³² bíh¹, hi³ jlánh¹ cue³² uu³mí²tsɨ³² la³ cun³ jmáɨ¹ hi³ ziáun² dí². La³jı̵́n³² ñí¹ jáh³ bíh¹ tɨn² tsá²mɨ³cuóun² ma³táɨn³², hi³ hí¹ la³ tɨ³ má²ca³ma³táɨn³² yáh³ tsú²; uá¹la³ jáh³ cánh¹, jáh³ ngɨ³² rón³² hué³², tan³² nı̵́², jɨ³ jáh³ jmáɨ² nı̵́² siáh³. Tɨ³la³ uá¹jinh¹ tɨn² tsáu² ma³táɨn³² jáh³, tɨ³la³ hí¹ jan² bíh¹ tiá² hin² tɨn² ma³táɨn³² zı̵́h¹ hnga². Zı̵́h¹ dí² jáun² dá² cáun² ti³ tɨn² jmu² hlaɨh³ bíh¹, la³ cun³ hi³ dí¹quian¹ hnga², hi³ jlánh¹ quian³² no¹ huáh² tsı̵́³ hi³ lı̵́¹³ jngah³. Quí¹ sa³ cun³quionh³ zı̵́h¹ jáun² bíh¹ dí² ma³quien¹³ dí² Dió³² Jméi² dí², hi³ cun³quionh³ zı̵́h¹ jáun² bíh¹ dí² siáh³ chú¹ juon¹ dí² tsá²mɨ³cuóun² raɨnh²¹ dí², tsá² lı̵́²ma³zian² hí³ Dió³² la³ cun³ rón³² lı̵́³ jáun² nóh³² quioh²¹ dí² hnga². Sa³ cáun² ho³ jáun² bíh¹ dí² hue³² jú¹ chu²¹ la³ má²quionh³ jú¹ hlah³. Hnoh² reh², tiá² jmı̵́¹ ca³tɨ²¹ cu³tí¹³ yáh³ hi³ lı̵́¹³ la³ jáun². ¿Hí¹ lı̵́¹³ bíh¹ hion¹³ jmɨ² cuóuh³ cu³tsa³² quionh³ jmɨ² ñeh¹ nı̵́¹juáh³ cáun² ñí¹ hion² jmáɨ² dúh¹? Hi³ sa³jun³ háɨ³² yáh³ huı̵́h² pih²¹ hmá² sí² co², sa³jun³ lı̵́¹³ ha³ siáh³ mɨ³ hmá² sí² co² hmáɨh³² uóun²jɨeh¹³. Reh², la³ jáun² bíh¹ lı̵́³ siáh³ hi³ hí¹ cónh³ yáh³ tiá² lı̵́¹³ hion¹³ jmɨ² cuóuh³ ñí¹ qui³ má²hion² jmɨ² ñeh¹. Hi³ jáun² né³, nı̵́¹juáh³ ja¹ quián¹³ hnoh² zian² jan² tsá² quia³lín³ jmı̵́¹ tsı̵́³, tsá² cháunh²³ re² chí¹, cuı̵́¹ jmu¹ tsú² lin¹ cun³quionh³ hi³ zian² tsú² hi³ chun¹, jɨ³ cun³quionh³ hi³ chu²¹ hi³ jmu² tsú². Tɨ³la³ la³ cun³ qui³ jmu² jan² tsá² quia³lín³ jmı̵́¹ tsı̵́³ má¹na²¹, hniáuh³² jmu³ tsú² hi³ tiá² quien² hnga² tá¹la³ jmu² tsú² jáun² hi³ chu²¹. Tɨ³la³ nı̵́¹juáh³ cuéh¹ hnoh² jáɨ¹³ hi³ lı̵́¹³ lı̵́n³ náh² tsá² ja³²lɨ³ uóu³² tsı̵́³, hi³ cáun² lı̵́¹ zian² náh² ca³tɨn¹ hmóu³², jáun² tiá² ca³tɨn¹ náh² jmúh¹³ tonh² hi³ jlánh¹ re² cháunh²³ honh² náh²; quí¹ má²ná¹jlı̵́h²³ bí¹ náh² jú¹ tson² cun³quionh³ jú¹ tı̵́¹jáɨ² nı̵́¹juáh³ la³ jáun². Quí¹ jun³juáh¹³ Dió³² yáh³ jmu² hi³ cháunh²³ chí¹ tsú² la³ nı̵́², hmóu³² bíh¹ tsá² zian² ñí¹ hué²¹ lá² hnauh² chí¹ la³ nı̵́². Cáun² hi³ tɨn² hmóu³² tsá²mɨ³cuóun² bíh¹ nɨ³², hi³ cháunh²³ nı̵́² chí¹ tsú² ja³² ñí¹con² tsá² hláɨnh¹ bíh¹. Quí¹ ñí¹ zian² tsá² uóu³² tsı̵́³, tsá² lı̵́¹ zian² jmáh³la³ hi³ ca³tɨn¹ hmóu³², jáun² ñí¹ la³ jáun² cáun² tiáunh¹ tsú² cú²tiú² cú²jan³² bíh¹, hi³ zia³² siáh³ la³jáh³ dú¹ ñí¹ hi³ hlah³. Tɨ³la³ la³ cun³ hi³ cháunh²³ jáun² tsı̵́³ tsú² hi³ ja³² ñí¹con² Dió³² má¹ná¹, lı̵́³ cáun² hi³ jɨ² bíh¹ la³ñí¹ la³ján³. Tsá² quia³lín³ jmı̵́¹ tsı̵́³ la³ hí³ bíh¹ lı̵́n³ jan² tsá² jmu² hi³ lı̵́¹³ niau²¹ tie³, jan² tsá² jmu² juenh² tsı̵́³, tsá² huá¹ chí¹, tsá² ja³² mií³ tsı̵́³, tsá² jmu² cá² ñí¹ hi³ chu²¹, tsá² jı̵́en³² tsáu² cú²re² he², tsá² tiá² zaɨ³² jë¹. Jáun² né³, tsá² má²re² hniéi² quioh²¹ tsáu², hi³ hnió³ hi³ zian² tsáu² cáun² hi³ re² hi³ tɨn², tsá² la³ hí³ bíh¹ jmu² hi³ tá²tsɨ²¹ héih³² cú²tso². ¿He³ láɨh³² zia³² hniéi² ja¹ quián¹³ hnoh², hi³ tiá² re² tiáunh¹ náh² cá²honh¹? ¿Haun¹³ juáh¹³ cun³ñí¹ hi³hliá² dí¹quiaunh²¹ honh² náh² hi³ hlah³ bíh¹, hi³ jáun² hú¹pı̵́² hú¹juoun³² honh² náh², quí¹ cun³ñí¹ hi³ zia³² hi³ hniá¹ náh²? Quí¹ zia³² bíh¹ hi³ hniá¹ hnoh² quioh²¹ tsáu², tɨ³la³ nı̵́¹juáh³ tiá² la³ ca³lɨ³ zia³² yáh³ hi³ jáun² quián¹³ hnoh² né³, jáun² cáun² jngɨh²³ bíh¹ náh² tsú². Hi³ nı̵́¹juáh³ tiá² ca³janh¹ hnoh² hi³ jmı̵́¹ hen² jáun² honh² náh² né³, jáun² cáun² lı̵́¹ ja³² uóu³² honh² bíh¹ náh², hi³ jmuh³² náh² hniéi², hi³ quiú² tɨn²³ náh² quiúnh¹ tsáu². Cun³ñí¹ hi³ tiá² mɨh³² ñí¹con² Dió³² bíh¹ náh², hi³ jáun² bíh¹ tiá² chanh¹ hnoh² la³ cun³ hi³ jmı̵́¹ hniá¹ náh² jáun². Hi³ uá¹jinh¹ mɨh³² náh², cun³ jáun² tiá² hián¹³ bíh¹ náh², quí¹ cun³ñí¹ hi³ hu²¹ siánh³ honh² náh² tá¹la³ mɨh³² náh² jáun², hi³ lı̵́¹ hnáuh² náh² má¹hiúnh¹³ honh² hmóu³² cun³quionh³ hi³ jmı̵́¹ hniá¹ náh² jáun² hian³. ¡Hnoh², tsá² lı̵́n³ náh² la³jmı̵́¹ lı̵́n³ tsá²mɨ³ juón¹! ¿Tiá¹ má²ñíh¹ hnoh² hi³ tsá² jéih³² hi³ zia³² ñí¹ hué²¹ lá², tsá² la³ hí³ má²lı̵́n³ jan² tsá² hon² Dió³² bíh¹? Quí¹ lɨ³ua³ jan² tsá² tɨ³² tsı̵́³ hi³ zia³² ñí¹ hué²¹ lá², tsá² la³ hí³ má²ná¹háun³ bíh¹ quionh³ Dió³². Quí¹ jun³juáh¹³ cáun² ti³ lı̵́¹ rá¹juáh³ yáh³ jáɨ¹³ quioh²¹ Dió³² ñí¹ rá¹juáh³ jáun² la³ lá²: “Jmı̵́²chí³ Chun¹ ca³tanh² hí³ Dió³² hñu³ tsı̵́³ dí² jlánh¹ hniau³ dí² ca³la³ hi³ hɨ³² lı̵́n³² tsı̵́³”. Hi³ jáun² né³, tɨ³ lɨ³mí¹ má²ca³ma³hé² ma³mieh² bíh¹ dí² Dió³², la³ cun³ rá¹juáh³ jáun² jáɨ¹³ quioh²¹ Dió³², hi³ juáh³ la³ lá²: “Dió³² tiá² cué³² jáɨ¹³ hi³ lı̵́¹³ lɨ³ quien² tsá² jmu² quien² hnga², tɨ³la³ má²hé² má²mieh² tsú² tsá² tsı̵́¹juı̵́³ bíh¹”. Jáun² né³, jɨenh²¹ náh² hmóu³² ñí¹con² Dió³²; hi³ jmu³ náh² huáh² chinh³² ñí¹con² tsá² hláɨnh¹, hi³ jáun² cuon³ tsú² ñí¹con² náh². Jmu³ náh² pí³ hi³ lɨ³ cuóun³² náh² Dió³², hi³ jáun² Dió³² lɨ³ cuóu³² siáh³ hnoh². Hnoh² tsá² ná¹ren² náh² tso³, ma³jı̵́¹³ náh² cuonh², hi³ jáun² lı̵́¹³ ná¹chan¹ ñí¹con² Dió³². Hi³ hnoh² né³, tsá² nio² tun³ honh², ma³jı̵́¹³ náh² honh², hi³ jáun² niau²¹ cáun² honh² náh². Chá¹ náh² hlaɨh³ honh², hi³ uo³ náh² ca³la³ hi³ jngɨh³² honh² náh². Cha³ jmı̵́¹ jú¹ jmı̵́¹ ngáɨh¹³ náh², cuı̵́¹ taɨn²¹ jmı̵́²zı̵́h¹ máh¹ náh²; hi³ cha³ jmı̵́¹ má¹hiúnh¹³ náh² honh² né³, cuı̵́¹ niau²¹ hlah³ honh² náh². Jmu³ náh² hi³ tsı̵́¹juı̵́³ náh² ta³ ñí¹ Dió³² Juo¹³ dí², jáun² hí³ né³ ziau³ chi³cuú² hi³ lɨ³ quien² náh². Hnoh² reh², tiú²uú² hniáuh³² ziú¹ hɨen¹³ náh² tsá²ján² tsá²ján²; quí¹ tsá² hleh³² hlah³ quioh²¹ raɨnh²¹, ho³ hi³ chú² ñih²³ siáh³ raɨnh²¹, tsá² la³ hí³ dá² hleh³² hlah³ ca³tɨ²¹ liei²¹ jáun² quioh²¹ Dió³² bíh¹, hi³ chú² ñih²³ tsú² siáh³ liei²¹ jáun². Quí¹ nı̵́¹juáh³ la³ chú¹ la³ ñih²¹ hnú² liei²¹ jáun², jáun² jmuh³² hnú² la³jmı̵́¹ jmu² jan² tsá² lı̵́n³ jue²¹ bíh¹ cha³ jmı̵́¹ má¹tih²¹ hnú² la³ cun³ hi³ juáh³ jáun² liei²¹. Quí¹ jan² tán¹ bíh¹ Tsá² ca³quiú² héih³² zian², hi³ lı̵́n³ siáh³ Jue²¹; hnga² hí³ bíh¹ siáh³ tɨn² lión³² tsáu², ho³lá²dá² hi³ hin³ tsú² tsáu². Hi³ jáun² né³ ¿hin² dá² tsánh² hnú², jáun² sa³ hnú² yáh³ hnáuh² jmúh¹³ jue²¹ hi³ rá¹tsɨh³² héih³² ñí¹con² tsá²mɨ³cuóun² renh²? Hi³ jáun² né³, hnoh² tsá² juáh³ la³ lá²: “Né³² ho³ tsa³háu² tsáu¹³ jnoh¹ cú²juú² cun³ cáun² mii², hi³ tsá¹hnáu¹³ jnoh¹ quɨe³”, niéi² náh² re² jáɨ¹³ lá²: ¡Hí¹juáh³ he³ lı̵́¹³ tsa³háu² yáh³ tiá² hi³ ñíh¹ hnoh², tiá¹³ bíh¹ tiá³ jlánh¹ ñíh¹ hnoh² hi³ he³ lı̵́¹³ jmı̵́¹tsú² jmı̵́¹ja³²! Quí¹ jmáɨ¹ hi³ zian² hnoh² jáun² dá² lı̵́³ la³jmı̵́¹ lı̵́³ cáun² jnie³ cháun¹ bíh¹, cáun² hi³ lı̵́¹ jnia² cu³tiá³ pih²¹, hi³ la³juɨ³² yein³² siáh³. La³ lá² bíh¹ jmı̵́¹ hniáuh²¹ juáh¹³ hnoh²: “Nı̵́¹juáh³ Dió³² Juo¹³ dí² hnió³, jáun² ziáun² bíh¹ dí² hi³ jmú¹³ dí² hi³ lá² ho³ hi³ ó³²”. Tɨ³la³ hnoh² má¹ná¹, cá² ñí¹ hléh¹ jú¹ tú² ráun³ bíh¹ hnoh² hi³ jmuh³² náh² tonh² lı̵́n³²; tɨ³la³ jáɨ¹³ la³ nı̵́² né³, hú¹tá¹ jú¹ hlah³ bíh¹. Hi³ jáun² nı̵́¹juáh³ zian² tsá² má²ñi³² he³ hi³ chu²¹ jmu³, tɨ³la³ má¹lɨ³² jáun² né³, tiá² hi³ jáun² jmu² yáh³ tsú², hi³ jáun² tsá² la³ hí³ má²ca³lɨ³ren² tso³ bíh¹. Ja³bí¹ hnoh² siáh³, tsá² hánh³, ¡niéi² náh² re² jáɨ¹³ lá²! Uo³ hnoh² hi³ tı̵́¹ hoh³ náh² hi³ ca³tɨ²¹ uu³cha³tsɨ³² hi³ né³bí¹ má¹tsoh¹ náh² honh². Lı̵́³ la³juah²¹ dúh¹ hi³ má²ca³cáh² bíh¹ la³jı̵́³² hi³ zia³² jáun² quián¹³ náh²; hi³ hmɨh³² chu²¹ hi³ nio³ náh² jáun² né³, lı̵́³ la³juah²¹ dúh¹ hi³ má²ca³cúh² mí¹cháu² bíh¹. La³jmı̵́¹ lı̵́³ hi³ má²ca³hiá² juóuh³² bíh¹ mí¹ñí² cú¹tiáu² jɨ³ mí¹ñí² cú¹mí¹niau²¹ jáun² quián¹³ hnoh². Juóuh³² jáun² né³ má²li²¹ hi³ tson² bíh¹ má²ca³lɨ³ren² náh² tso³, hi³ jáun² cun³quionh³ juóuh³² jáun² bíh¹ cón³² hnoh² la³juah²¹ dúh¹ hi³ hún¹ sı̵́². Quí¹ jmáɨ¹ hi³ tiauh² dí² lá² jlánh¹ má²ca³cháh¹ hnoh² cu³lɨ²¹ hi³ lı̵́¹³ lɨ³ hánh³ náh². Sá¹nı̵́² niéi² náh², jɨe³ cónh³ tso³ tiá² cha³² quɨe³ hi³ jmı̵́¹ hniáuh²¹ má¹hmah²¹ náh² ñí¹con² tsá² ca³jmú³ ta²¹ ñí¹náɨ² quián¹³ náh²; hi³ Dió³² Tsá² lı̵́n³ Juo¹³ hliáu³ tionh² hñu³mɨ³cuú² né³, má²ca³náɨ³² jáɨ¹³ hi³ tú² hna² jáun² tsá² ca³jmú³ hí³ ta²¹ ñí¹con² náh². Jlánh¹ re² ma²ca³ma³zián¹³ hnoh² ñí¹ hué²¹ lá², hi³ cáun² lı̵́¹ má²ca³ma³hiúnh¹³ hnoh² honh² hi³ jmúh¹³ náh² lɨ³ua³ cáun² hi³ ca³lɨ³ hniá¹ náh² jmúh¹³. ¡La³jmı̵́¹ má²cánh¹ tsú² cuá¹juı̵́² hi³ má²jngɨh³ bíh¹ tsú², má²ca³ma³hion²¹ náh² honh²! Hi³ cáun² lı̵́¹ ca³ra³can³² náh² tso³ tsá² tiá² tso³ ren², hi³ ca³jngɨh³² náh²; uá¹jinh¹ tsá² tiá² hi³ hlah³ ca³jmú³ hí¹ cáun² ñí¹con² hnoh². Hi³ jáun² né³ reh², tsá² ná¹janh³² náh² hi³ jáunh³ Tɨ³² Juo¹³ dí², hniáuh³² ná¹hu²¹ honh² náh² ca³tɨ²¹ jmáɨ¹ jáun² la³jmı̵́¹ ná¹hu²¹ tsı̵́³ tsá² zia³² hi³ má²jná¹ ñí¹náɨ² quioh²¹, tsá² ná¹hé² ná¹jan³² hi³ chau¹³ jmı̵́³ la³ cun³ jmáɨ¹ hi³ lɨ³ hniáuh³², quí¹ hu²¹ tsı̵́³ tsú² hi³ ló³² re² hi³ quioh²¹. Jáun² né³, ja³bí¹ hnoh² siáh³ hniáuh³² cháh¹³ náh² tiá³ honh² ca³tɨ²¹ hi³ hu²¹ jáun² honh² náh² hi³ má²ja³quián³ jáunh³ Tɨ³² Juo¹³ dí². Hnoh² reh², ha³ lı̵́² chú¹ lı̵́² ñih²¹ náh² tsá²ján² tsá²ján², jáun² tiá² tá¹tsɨn¹ náh² héih³², quí¹ má²ja³quián³ cu³tí¹³ bíh¹ jáunh³ Dió³² Tsá² lı̵́n³ Jue²¹. Hi³ jáun² né³ reh², chú³² jmu³ náh² la³jmı̵́¹ ca³jmú³ jáun² la³jı̵́n³² tsá² ca³lɨn³ hí³ *tɨ³² jë¹ Dió³², tsá² ca³hléh³ hí³ cha¹³ Tɨ³² Juo¹³ dí²; uá¹jinh¹ ca³ma³tso² lı̵́n³² tsú² tsı̵́³, tɨ³la³ ca³tiánh³ bíh¹ tsú² hi³ ná¹hu²¹ cáun² tsı̵́³. Sá¹nı̵́² jɨe³, jnoh¹ ná¹láɨ²³ hi³ jlánh¹ jmı̵́¹ ren² tsá² ca³ma³tso² hí³ tsı̵́³. Uá¹la³ cun³ hi³ má²ca³niéih² náh² jáun² hi³ ca³tɨn¹ hla¹ Job; tsá² hí³ jlánh¹ re² ca³cueh³ tsı̵́³ ñí¹ ca³la³ jı̵́³² hi³ ca³quiúnh³², hi³ ná¹ñíh¹ náh² siáh³ he³ ca³hiauh³ tsú² ñí¹con² Dió³² jmı̵́¹ lɨ²¹ jáun², quí¹ Dió³² jlánh¹ chun¹ hi³ ja³² lı̵́n³² mií³ tsı̵́³. Jáun² né³ reh², zia³² bíh¹ siáh³ cáun² jú¹ tson² hi³ jlánh¹ bíh¹ quien² hi³ juáh³ la³ lá²: Tiá² hniáuh³² má¹quien¹³ náh² jáɨ¹³ quián¹³ cun³quionh³ hi³ hɨen¹³ náh² Dió³², sa³jun³ hi³ hɨen¹³ náh² tsá² zian² ñí¹ hué²¹ lá² siáh³, sa³jun³ hí¹ cáun² hi³ siáh³ siáh³ tiá² hniáuh³² hı̵́eh¹ náh² hi³ má¹quien¹³ náh² jáɨ¹³ quián¹³. Quí¹ jáɨ¹³ la³ nı̵́² tiá² lin¹ lɨ³ hniáuh³² yáh³ nı̵́¹juáh³ hi³ ná¹lı̵́n³ hnoh² tsá² tson² jëh² náh², uá¹ hi³ ca³juah²¹ náh²: “Tson²”, uá¹ hi³ ca³juah²¹ náh²: “Tiá² tson²”; quí¹ nı̵́² lı̵́¹ ca³ta³zanh¹ náh² ñí¹ hlah³. Hi³ jáun² né³, nı̵́¹juáh³ ja¹ quián¹³ hnoh² zian² jan² tsá² má²tso² tsı̵́³, cuı̵́¹ liéinh²¹ tsú² Dió³². Hi³ nı̵́¹juáh³ zian² tsá² re² nio² tsı̵́³, jáun² cuı̵́¹ má¹quien² tsú² Dió³² cun³quionh³ hi³ hɨe³² tsú² jáun² sun¹. Hi³ nı̵́¹juáh³ zian² jan² tsá² tsáun¹, cuı̵́¹ tiéh¹³ tsú² tsá²daun³² tsá² ná¹ñí¹ cuáh³², jáun² tsá² hí³ tsa³táunh¹ hi³ tsa³lienh³ Dió³² cha¹³ Tɨ³² Juo¹³ dí² hi³ ca³tɨn¹ tsú², hi³ jñéi³ tsú² siáh³ no¹ chí¹ tsú². Jáun² nı̵́¹ má¹ca³liéinh³² tsú² jáun² Dió³² hi³ ná¹tioh³ cáun² tsı̵́³ tsú², lan¹³ bíh¹ tsá² tsáun¹. Dió³² bíh¹ jmah³ tsá² hí³; hi³ nı̵́¹juáh³ ren² tsú² tso³, ja³bí¹ hin³ siáh³ tsáu¹³ tsú² uá²jaɨ³². Jáun² né³, cun³ñí¹ hi³ jáun² bíh¹ chú³² ton¹³ náh² tsáuh³ ñí¹con² tsá²ján² tsá²ján², hi³ lienh¹ náh² siáh³ Dió³² ca³tɨn¹ tsá²ján² tsá²ján², jáun² lan¹³ náh². Nı̵́¹juáh³ jan² tsá² zian² la³ cun³ hi³ tɨ³² tsı̵́³ Dió³² liéinh³² Dió³² \textsquarebracketleft{}ca³la³ jonh³ jmı̵́¹ tsı̵́³\textsquarebracketright{}, hú¹tá¹ cu³tí³ jmu³ bíh¹ ta²¹. Uá¹la³ cun³ hi³ ca³lɨ³ jáun² ñí¹con² Líh³, tsá² jmı̵́¹ lı̵́n³ hí³ tɨ³² jë¹ Dió³² jmı̵́¹tin². Ja³bí¹ tsá² hí³ siáh³ jmı̵́¹ lı̵́n³ jan² tsá² lı̵́¹ tsáu² la³ jnoh¹ bíh¹; tɨ³la³ jmı̵́¹ ca³liéinh³² tsú² jáun² Dió³² né³, hi³ ca³mı̵́³ tsú² hi³ tiú²uú² chau¹³ jmı̵́³, jáun² ca³quin³ bíh¹ jmı̵́³ cun³ hnɨ³² mii² tón³² hué³² ñí¹ jmı̵́¹ cuá³ tsú² jáun². Jmı̵́¹ lɨ²¹ jáun² né³, ca³liéinh³² tsú² siáh³ Dió³², jáun² ca³jauh³ bíh¹ siáh³ jmı̵́³ hué³² jáun², jáun² ca³cuú² ca³láu² bíh¹ re² la³jı̵́³² hi³ lau³ cuá¹ hué²¹. Hnoh² reh², nı̵́¹juáh³ ja¹ quián¹³ hnoh² zian² jan² tsá² cón³² siáh³ jan² tsá² jmı̵́¹ má²ngau³² tɨ³ có³² ca³tɨ²¹ jú¹ tson², jáun² cháu¹ náh² honh² hi³ tsá² jmu² ta²¹ la³ jáun² má²ca³lión³² jan² tsá² jmı̵́¹ má²jún¹ bíh¹, hi³ jmu² tsú² siáh³ hi³ lı̵́¹³ hin³ ca³la³ jı̵́³² tso³ hi³ jmı̵́¹ ren² jáun² tsá² hí³. \textsquarebracketleft{}Cun³ nı̵́² bíh¹ tí³ jáɨ¹³ hi³ juanh³² jná¹³ ñí¹con² hnoh² reh².\textsquarebracketright{}}
\end{description}
\clearpage
{\clearpage
\thispagestyle{bodyfirstpage}\vspace*{.65in}\noindent
\raisebox{\baselineskip}[0pt]{\protect\hypertarget{rXLingPapGlossary1}{}}\raisebox{\baselineskip}[0pt]{\pdfbookmark[1]{Appendex I: Glossary of technical concepts and terms}{rXLingPapGlossary1}}{\MakeUppercase{{\protect\centering
Appendex I: Glossary of technical concepts and terms\\}}}\XLingPaperaddtocontents{rXLingPapGlossary1}\markboth{Appendex I: Glossary of technical concepts and terms}{Appendex I: Glossary of technical concepts and terms}}\par{}
\vspace{10.8pt}\indent Some technical terms and abbreviations used in this paper\par{}\needspace{5\baselineskip}

\penalty-3000
\begin{description}
\setlength{\topsep}{0pt}\setlength{\partopsep}{0pt}\setlength{\itemsep}{0pt}\setlength{\parsep}{0pt}\setlength{\parskip}{0pt}\setlength{\leftmargini}{1em}\setlength{\leftmarginii}{1em}\setlength{\leftmarginiii}{1em}\setlength{\leftmarginiv}{1em}\penalty10000\item[Orthography]{a writing system for a given language.}
\penalty10000\item[Writing system]{an implementation of one or more scripts to form a complete system for writing a particular language. http://scripts.sil.org/cms/scripts/page.php?cat\_id=Glossary\#writingsys}
\penalty10000\item[Human-Computer Interation (HCI)]{}
\end{description}
\XLingPaperendtableofcontents
\pagebreak\end{MainFont}
\end{document}
